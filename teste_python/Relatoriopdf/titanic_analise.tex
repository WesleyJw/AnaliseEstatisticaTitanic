
% Default to the notebook output style

    


% Inherit from the specified cell style.




    
\documentclass[11pt]{article}

    
    
    \usepackage[T1]{fontenc}
    % Nicer default font (+ math font) than Computer Modern for most use cases
    \usepackage{mathpazo}

    % Basic figure setup, for now with no caption control since it's done
    % automatically by Pandoc (which extracts ![](path) syntax from Markdown).
    \usepackage{graphicx}
    % We will generate all images so they have a width \maxwidth. This means
    % that they will get their normal width if they fit onto the page, but
    % are scaled down if they would overflow the margins.
    \makeatletter
    \def\maxwidth{\ifdim\Gin@nat@width>\linewidth\linewidth
    \else\Gin@nat@width\fi}
    \makeatother
    \let\Oldincludegraphics\includegraphics
    % Set max figure width to be 80% of text width, for now hardcoded.
    \renewcommand{\includegraphics}[1]{\Oldincludegraphics[width=.8\maxwidth]{#1}}
    % Ensure that by default, figures have no caption (until we provide a
    % proper Figure object with a Caption API and a way to capture that
    % in the conversion process - todo).
    \usepackage{caption}
    \DeclareCaptionLabelFormat{nolabel}{}
    \captionsetup{labelformat=nolabel}

    \usepackage{adjustbox} % Used to constrain images to a maximum size 
    \usepackage{xcolor} % Allow colors to be defined
    \usepackage{enumerate} % Needed for markdown enumerations to work
    \usepackage{geometry} % Used to adjust the document margins
    \usepackage{amsmath} % Equations
    \usepackage{amssymb} % Equations
    \usepackage{textcomp} % defines textquotesingle
    % Hack from http://tex.stackexchange.com/a/47451/13684:
    \AtBeginDocument{%
        \def\PYZsq{\textquotesingle}% Upright quotes in Pygmentized code
    }
    \usepackage{upquote} % Upright quotes for verbatim code
    \usepackage{eurosym} % defines \euro
    \usepackage[mathletters]{ucs} % Extended unicode (utf-8) support
    \usepackage[utf8x]{inputenc} % Allow utf-8 characters in the tex document
    \usepackage{fancyvrb} % verbatim replacement that allows latex
    \usepackage{grffile} % extends the file name processing of package graphics 
                         % to support a larger range 
    % The hyperref package gives us a pdf with properly built
    % internal navigation ('pdf bookmarks' for the table of contents,
    % internal cross-reference links, web links for URLs, etc.)
    \usepackage{hyperref}
    \usepackage{longtable} % longtable support required by pandoc >1.10
    \usepackage{booktabs}  % table support for pandoc > 1.12.2
    \usepackage[inline]{enumitem} % IRkernel/repr support (it uses the enumerate* environment)
    \usepackage[normalem]{ulem} % ulem is needed to support strikethroughs (\sout)
                                % normalem makes italics be italics, not underlines
    

    
    
    % Colors for the hyperref package
    \definecolor{urlcolor}{rgb}{0,.145,.698}
    \definecolor{linkcolor}{rgb}{.71,0.21,0.01}
    \definecolor{citecolor}{rgb}{.12,.54,.11}

    % ANSI colors
    \definecolor{ansi-black}{HTML}{3E424D}
    \definecolor{ansi-black-intense}{HTML}{282C36}
    \definecolor{ansi-red}{HTML}{E75C58}
    \definecolor{ansi-red-intense}{HTML}{B22B31}
    \definecolor{ansi-green}{HTML}{00A250}
    \definecolor{ansi-green-intense}{HTML}{007427}
    \definecolor{ansi-yellow}{HTML}{DDB62B}
    \definecolor{ansi-yellow-intense}{HTML}{B27D12}
    \definecolor{ansi-blue}{HTML}{208FFB}
    \definecolor{ansi-blue-intense}{HTML}{0065CA}
    \definecolor{ansi-magenta}{HTML}{D160C4}
    \definecolor{ansi-magenta-intense}{HTML}{A03196}
    \definecolor{ansi-cyan}{HTML}{60C6C8}
    \definecolor{ansi-cyan-intense}{HTML}{258F8F}
    \definecolor{ansi-white}{HTML}{C5C1B4}
    \definecolor{ansi-white-intense}{HTML}{A1A6B2}

    % commands and environments needed by pandoc snippets
    % extracted from the output of `pandoc -s`
    \providecommand{\tightlist}{%
      \setlength{\itemsep}{0pt}\setlength{\parskip}{0pt}}
    \DefineVerbatimEnvironment{Highlighting}{Verbatim}{commandchars=\\\{\}}
    % Add ',fontsize=\small' for more characters per line
    \newenvironment{Shaded}{}{}
    \newcommand{\KeywordTok}[1]{\textcolor[rgb]{0.00,0.44,0.13}{\textbf{{#1}}}}
    \newcommand{\DataTypeTok}[1]{\textcolor[rgb]{0.56,0.13,0.00}{{#1}}}
    \newcommand{\DecValTok}[1]{\textcolor[rgb]{0.25,0.63,0.44}{{#1}}}
    \newcommand{\BaseNTok}[1]{\textcolor[rgb]{0.25,0.63,0.44}{{#1}}}
    \newcommand{\FloatTok}[1]{\textcolor[rgb]{0.25,0.63,0.44}{{#1}}}
    \newcommand{\CharTok}[1]{\textcolor[rgb]{0.25,0.44,0.63}{{#1}}}
    \newcommand{\StringTok}[1]{\textcolor[rgb]{0.25,0.44,0.63}{{#1}}}
    \newcommand{\CommentTok}[1]{\textcolor[rgb]{0.38,0.63,0.69}{\textit{{#1}}}}
    \newcommand{\OtherTok}[1]{\textcolor[rgb]{0.00,0.44,0.13}{{#1}}}
    \newcommand{\AlertTok}[1]{\textcolor[rgb]{1.00,0.00,0.00}{\textbf{{#1}}}}
    \newcommand{\FunctionTok}[1]{\textcolor[rgb]{0.02,0.16,0.49}{{#1}}}
    \newcommand{\RegionMarkerTok}[1]{{#1}}
    \newcommand{\ErrorTok}[1]{\textcolor[rgb]{1.00,0.00,0.00}{\textbf{{#1}}}}
    \newcommand{\NormalTok}[1]{{#1}}
    
    % Additional commands for more recent versions of Pandoc
    \newcommand{\ConstantTok}[1]{\textcolor[rgb]{0.53,0.00,0.00}{{#1}}}
    \newcommand{\SpecialCharTok}[1]{\textcolor[rgb]{0.25,0.44,0.63}{{#1}}}
    \newcommand{\VerbatimStringTok}[1]{\textcolor[rgb]{0.25,0.44,0.63}{{#1}}}
    \newcommand{\SpecialStringTok}[1]{\textcolor[rgb]{0.73,0.40,0.53}{{#1}}}
    \newcommand{\ImportTok}[1]{{#1}}
    \newcommand{\DocumentationTok}[1]{\textcolor[rgb]{0.73,0.13,0.13}{\textit{{#1}}}}
    \newcommand{\AnnotationTok}[1]{\textcolor[rgb]{0.38,0.63,0.69}{\textbf{\textit{{#1}}}}}
    \newcommand{\CommentVarTok}[1]{\textcolor[rgb]{0.38,0.63,0.69}{\textbf{\textit{{#1}}}}}
    \newcommand{\VariableTok}[1]{\textcolor[rgb]{0.10,0.09,0.49}{{#1}}}
    \newcommand{\ControlFlowTok}[1]{\textcolor[rgb]{0.00,0.44,0.13}{\textbf{{#1}}}}
    \newcommand{\OperatorTok}[1]{\textcolor[rgb]{0.40,0.40,0.40}{{#1}}}
    \newcommand{\BuiltInTok}[1]{{#1}}
    \newcommand{\ExtensionTok}[1]{{#1}}
    \newcommand{\PreprocessorTok}[1]{\textcolor[rgb]{0.74,0.48,0.00}{{#1}}}
    \newcommand{\AttributeTok}[1]{\textcolor[rgb]{0.49,0.56,0.16}{{#1}}}
    \newcommand{\InformationTok}[1]{\textcolor[rgb]{0.38,0.63,0.69}{\textbf{\textit{{#1}}}}}
    \newcommand{\WarningTok}[1]{\textcolor[rgb]{0.38,0.63,0.69}{\textbf{\textit{{#1}}}}}
    
    
    % Define a nice break command that doesn't care if a line doesn't already
    % exist.
    \def\br{\hspace*{\fill} \\* }
    % Math Jax compatability definitions
    \def\gt{>}
    \def\lt{<}
    % Document parameters
    \title{Análise Estatística - Dados Titanic\\
    	José Wesley Lima Silva}
 
    
    
    

    % Pygments definitions
    
\makeatletter
\def\PY@reset{\let\PY@it=\relax \let\PY@bf=\relax%
    \let\PY@ul=\relax \let\PY@tc=\relax%
    \let\PY@bc=\relax \let\PY@ff=\relax}
\def\PY@tok#1{\csname PY@tok@#1\endcsname}
\def\PY@toks#1+{\ifx\relax#1\empty\else%
    \PY@tok{#1}\expandafter\PY@toks\fi}
\def\PY@do#1{\PY@bc{\PY@tc{\PY@ul{%
    \PY@it{\PY@bf{\PY@ff{#1}}}}}}}
\def\PY#1#2{\PY@reset\PY@toks#1+\relax+\PY@do{#2}}

\expandafter\def\csname PY@tok@mh\endcsname{\def\PY@tc##1{\textcolor[rgb]{0.40,0.40,0.40}{##1}}}
\expandafter\def\csname PY@tok@il\endcsname{\def\PY@tc##1{\textcolor[rgb]{0.40,0.40,0.40}{##1}}}
\expandafter\def\csname PY@tok@nb\endcsname{\def\PY@tc##1{\textcolor[rgb]{0.00,0.50,0.00}{##1}}}
\expandafter\def\csname PY@tok@nd\endcsname{\def\PY@tc##1{\textcolor[rgb]{0.67,0.13,1.00}{##1}}}
\expandafter\def\csname PY@tok@gr\endcsname{\def\PY@tc##1{\textcolor[rgb]{1.00,0.00,0.00}{##1}}}
\expandafter\def\csname PY@tok@cp\endcsname{\def\PY@tc##1{\textcolor[rgb]{0.74,0.48,0.00}{##1}}}
\expandafter\def\csname PY@tok@ch\endcsname{\let\PY@it=\textit\def\PY@tc##1{\textcolor[rgb]{0.25,0.50,0.50}{##1}}}
\expandafter\def\csname PY@tok@sc\endcsname{\def\PY@tc##1{\textcolor[rgb]{0.73,0.13,0.13}{##1}}}
\expandafter\def\csname PY@tok@sb\endcsname{\def\PY@tc##1{\textcolor[rgb]{0.73,0.13,0.13}{##1}}}
\expandafter\def\csname PY@tok@w\endcsname{\def\PY@tc##1{\textcolor[rgb]{0.73,0.73,0.73}{##1}}}
\expandafter\def\csname PY@tok@gp\endcsname{\let\PY@bf=\textbf\def\PY@tc##1{\textcolor[rgb]{0.00,0.00,0.50}{##1}}}
\expandafter\def\csname PY@tok@cm\endcsname{\let\PY@it=\textit\def\PY@tc##1{\textcolor[rgb]{0.25,0.50,0.50}{##1}}}
\expandafter\def\csname PY@tok@nv\endcsname{\def\PY@tc##1{\textcolor[rgb]{0.10,0.09,0.49}{##1}}}
\expandafter\def\csname PY@tok@gu\endcsname{\let\PY@bf=\textbf\def\PY@tc##1{\textcolor[rgb]{0.50,0.00,0.50}{##1}}}
\expandafter\def\csname PY@tok@si\endcsname{\let\PY@bf=\textbf\def\PY@tc##1{\textcolor[rgb]{0.73,0.40,0.53}{##1}}}
\expandafter\def\csname PY@tok@gs\endcsname{\let\PY@bf=\textbf}
\expandafter\def\csname PY@tok@mo\endcsname{\def\PY@tc##1{\textcolor[rgb]{0.40,0.40,0.40}{##1}}}
\expandafter\def\csname PY@tok@nl\endcsname{\def\PY@tc##1{\textcolor[rgb]{0.63,0.63,0.00}{##1}}}
\expandafter\def\csname PY@tok@go\endcsname{\def\PY@tc##1{\textcolor[rgb]{0.53,0.53,0.53}{##1}}}
\expandafter\def\csname PY@tok@kp\endcsname{\def\PY@tc##1{\textcolor[rgb]{0.00,0.50,0.00}{##1}}}
\expandafter\def\csname PY@tok@c1\endcsname{\let\PY@it=\textit\def\PY@tc##1{\textcolor[rgb]{0.25,0.50,0.50}{##1}}}
\expandafter\def\csname PY@tok@cs\endcsname{\let\PY@it=\textit\def\PY@tc##1{\textcolor[rgb]{0.25,0.50,0.50}{##1}}}
\expandafter\def\csname PY@tok@err\endcsname{\def\PY@bc##1{\setlength{\fboxsep}{0pt}\fcolorbox[rgb]{1.00,0.00,0.00}{1,1,1}{\strut ##1}}}
\expandafter\def\csname PY@tok@cpf\endcsname{\let\PY@it=\textit\def\PY@tc##1{\textcolor[rgb]{0.25,0.50,0.50}{##1}}}
\expandafter\def\csname PY@tok@sx\endcsname{\def\PY@tc##1{\textcolor[rgb]{0.00,0.50,0.00}{##1}}}
\expandafter\def\csname PY@tok@nc\endcsname{\let\PY@bf=\textbf\def\PY@tc##1{\textcolor[rgb]{0.00,0.00,1.00}{##1}}}
\expandafter\def\csname PY@tok@vg\endcsname{\def\PY@tc##1{\textcolor[rgb]{0.10,0.09,0.49}{##1}}}
\expandafter\def\csname PY@tok@nn\endcsname{\let\PY@bf=\textbf\def\PY@tc##1{\textcolor[rgb]{0.00,0.00,1.00}{##1}}}
\expandafter\def\csname PY@tok@sa\endcsname{\def\PY@tc##1{\textcolor[rgb]{0.73,0.13,0.13}{##1}}}
\expandafter\def\csname PY@tok@ni\endcsname{\let\PY@bf=\textbf\def\PY@tc##1{\textcolor[rgb]{0.60,0.60,0.60}{##1}}}
\expandafter\def\csname PY@tok@vi\endcsname{\def\PY@tc##1{\textcolor[rgb]{0.10,0.09,0.49}{##1}}}
\expandafter\def\csname PY@tok@mf\endcsname{\def\PY@tc##1{\textcolor[rgb]{0.40,0.40,0.40}{##1}}}
\expandafter\def\csname PY@tok@c\endcsname{\let\PY@it=\textit\def\PY@tc##1{\textcolor[rgb]{0.25,0.50,0.50}{##1}}}
\expandafter\def\csname PY@tok@k\endcsname{\let\PY@bf=\textbf\def\PY@tc##1{\textcolor[rgb]{0.00,0.50,0.00}{##1}}}
\expandafter\def\csname PY@tok@mb\endcsname{\def\PY@tc##1{\textcolor[rgb]{0.40,0.40,0.40}{##1}}}
\expandafter\def\csname PY@tok@kn\endcsname{\let\PY@bf=\textbf\def\PY@tc##1{\textcolor[rgb]{0.00,0.50,0.00}{##1}}}
\expandafter\def\csname PY@tok@vm\endcsname{\def\PY@tc##1{\textcolor[rgb]{0.10,0.09,0.49}{##1}}}
\expandafter\def\csname PY@tok@bp\endcsname{\def\PY@tc##1{\textcolor[rgb]{0.00,0.50,0.00}{##1}}}
\expandafter\def\csname PY@tok@ss\endcsname{\def\PY@tc##1{\textcolor[rgb]{0.10,0.09,0.49}{##1}}}
\expandafter\def\csname PY@tok@nf\endcsname{\def\PY@tc##1{\textcolor[rgb]{0.00,0.00,1.00}{##1}}}
\expandafter\def\csname PY@tok@kr\endcsname{\let\PY@bf=\textbf\def\PY@tc##1{\textcolor[rgb]{0.00,0.50,0.00}{##1}}}
\expandafter\def\csname PY@tok@kd\endcsname{\let\PY@bf=\textbf\def\PY@tc##1{\textcolor[rgb]{0.00,0.50,0.00}{##1}}}
\expandafter\def\csname PY@tok@kt\endcsname{\def\PY@tc##1{\textcolor[rgb]{0.69,0.00,0.25}{##1}}}
\expandafter\def\csname PY@tok@s2\endcsname{\def\PY@tc##1{\textcolor[rgb]{0.73,0.13,0.13}{##1}}}
\expandafter\def\csname PY@tok@gt\endcsname{\def\PY@tc##1{\textcolor[rgb]{0.00,0.27,0.87}{##1}}}
\expandafter\def\csname PY@tok@no\endcsname{\def\PY@tc##1{\textcolor[rgb]{0.53,0.00,0.00}{##1}}}
\expandafter\def\csname PY@tok@gi\endcsname{\def\PY@tc##1{\textcolor[rgb]{0.00,0.63,0.00}{##1}}}
\expandafter\def\csname PY@tok@o\endcsname{\def\PY@tc##1{\textcolor[rgb]{0.40,0.40,0.40}{##1}}}
\expandafter\def\csname PY@tok@fm\endcsname{\def\PY@tc##1{\textcolor[rgb]{0.00,0.00,1.00}{##1}}}
\expandafter\def\csname PY@tok@ow\endcsname{\let\PY@bf=\textbf\def\PY@tc##1{\textcolor[rgb]{0.67,0.13,1.00}{##1}}}
\expandafter\def\csname PY@tok@ne\endcsname{\let\PY@bf=\textbf\def\PY@tc##1{\textcolor[rgb]{0.82,0.25,0.23}{##1}}}
\expandafter\def\csname PY@tok@nt\endcsname{\let\PY@bf=\textbf\def\PY@tc##1{\textcolor[rgb]{0.00,0.50,0.00}{##1}}}
\expandafter\def\csname PY@tok@gh\endcsname{\let\PY@bf=\textbf\def\PY@tc##1{\textcolor[rgb]{0.00,0.00,0.50}{##1}}}
\expandafter\def\csname PY@tok@mi\endcsname{\def\PY@tc##1{\textcolor[rgb]{0.40,0.40,0.40}{##1}}}
\expandafter\def\csname PY@tok@vc\endcsname{\def\PY@tc##1{\textcolor[rgb]{0.10,0.09,0.49}{##1}}}
\expandafter\def\csname PY@tok@sh\endcsname{\def\PY@tc##1{\textcolor[rgb]{0.73,0.13,0.13}{##1}}}
\expandafter\def\csname PY@tok@na\endcsname{\def\PY@tc##1{\textcolor[rgb]{0.49,0.56,0.16}{##1}}}
\expandafter\def\csname PY@tok@s\endcsname{\def\PY@tc##1{\textcolor[rgb]{0.73,0.13,0.13}{##1}}}
\expandafter\def\csname PY@tok@sr\endcsname{\def\PY@tc##1{\textcolor[rgb]{0.73,0.40,0.53}{##1}}}
\expandafter\def\csname PY@tok@gd\endcsname{\def\PY@tc##1{\textcolor[rgb]{0.63,0.00,0.00}{##1}}}
\expandafter\def\csname PY@tok@m\endcsname{\def\PY@tc##1{\textcolor[rgb]{0.40,0.40,0.40}{##1}}}
\expandafter\def\csname PY@tok@kc\endcsname{\let\PY@bf=\textbf\def\PY@tc##1{\textcolor[rgb]{0.00,0.50,0.00}{##1}}}
\expandafter\def\csname PY@tok@sd\endcsname{\let\PY@it=\textit\def\PY@tc##1{\textcolor[rgb]{0.73,0.13,0.13}{##1}}}
\expandafter\def\csname PY@tok@dl\endcsname{\def\PY@tc##1{\textcolor[rgb]{0.73,0.13,0.13}{##1}}}
\expandafter\def\csname PY@tok@se\endcsname{\let\PY@bf=\textbf\def\PY@tc##1{\textcolor[rgb]{0.73,0.40,0.13}{##1}}}
\expandafter\def\csname PY@tok@ge\endcsname{\let\PY@it=\textit}
\expandafter\def\csname PY@tok@s1\endcsname{\def\PY@tc##1{\textcolor[rgb]{0.73,0.13,0.13}{##1}}}

\def\PYZbs{\char`\\}
\def\PYZus{\char`\_}
\def\PYZob{\char`\{}
\def\PYZcb{\char`\}}
\def\PYZca{\char`\^}
\def\PYZam{\char`\&}
\def\PYZlt{\char`\<}
\def\PYZgt{\char`\>}
\def\PYZsh{\char`\#}
\def\PYZpc{\char`\%}
\def\PYZdl{\char`\$}
\def\PYZhy{\char`\-}
\def\PYZsq{\char`\'}
\def\PYZdq{\char`\"}
\def\PYZti{\char`\~}
% for compatibility with earlier versions
\def\PYZat{@}
\def\PYZlb{[}
\def\PYZrb{]}
\makeatother


    % Exact colors from NB
    \definecolor{incolor}{rgb}{0.0, 0.0, 0.5}
    \definecolor{outcolor}{rgb}{0.545, 0.0, 0.0}



    
    % Prevent overflowing lines due to hard-to-break entities
    \sloppy 
    % Setup hyperref package
    \hypersetup{
      breaklinks=true,  % so long urls are correctly broken across lines
      colorlinks=true,
      urlcolor=urlcolor,
      linkcolor=linkcolor,
      citecolor=citecolor,
      }
    % Slightly bigger margins than the latex defaults
    
    \geometry{verbose,tmargin=1in,bmargin=1in,lmargin=1in,rmargin=1in}
    
    

    \begin{document}
    
    
    \maketitle
    
    

    
    \section{Análise Estatística - Dados
Titanic}\label{anuxe1lise-estatuxedstica---dados-titanic}

    \subsection{Explorando o problema}\label{explorando-o-problema}

    \begin{Verbatim}[commandchars=\\\{\}]
{\color{incolor}In [{\color{incolor}1}]:} \PY{c+c1}{\PYZsh{}JWLS 20/07}
        \PY{c+c1}{\PYZsh{}\PYZsh{}Bibliotecas python necessárias}
        \PY{c+c1}{\PYZsh{}Manipulacao de dados}
        \PY{k+kn}{import} \PY{n+nn}{pandas} \PY{k}{as} \PY{n+nn}{pd}
        
        \PY{c+c1}{\PYZsh{}Testes e recursos estatisticos}
        \PY{k+kn}{import} \PY{n+nn}{numpy} \PY{k}{as} \PY{n+nn}{np}
        \PY{k+kn}{from} \PY{n+nn}{scipy} \PY{k}{import} \PY{n}{stats}
        
        \PY{c+c1}{\PYZsh{}Analise grafica}
        \PY{k+kn}{import} \PY{n+nn}{matplotlib}\PY{n+nn}{.}\PY{n+nn}{pyplot} \PY{k}{as} \PY{n+nn}{plt}
        \PY{k+kn}{import} \PY{n+nn}{matplotlib}\PY{n+nn}{.}\PY{n+nn}{lines} \PY{k}{as} \PY{n+nn}{mlines}
        \PY{k+kn}{import} \PY{n+nn}{seaborn} \PY{k}{as} \PY{n+nn}{sns}
        \PY{o}{\PYZpc{}}\PY{k}{matplotlib} inline
        
        \PY{c+c1}{\PYZsh{}Modelos, metricas e Machine Learning}
        \PY{k+kn}{from} \PY{n+nn}{sklearn}\PY{n+nn}{.}\PY{n+nn}{preprocessing} \PY{k}{import} \PY{n}{LabelEncoder}\PY{p}{,} \PY{n}{OneHotEncoder}
        \PY{k+kn}{from} \PY{n+nn}{sklearn}\PY{n+nn}{.}\PY{n+nn}{preprocessing} \PY{k}{import} \PY{n}{StandardScaler}
        \PY{k+kn}{from} \PY{n+nn}{sklearn}\PY{n+nn}{.}\PY{n+nn}{linear\PYZus{}model} \PY{k}{import} \PY{n}{LinearRegression}\PY{p}{,}\PY{n}{LogisticRegression}
        \PY{k+kn}{from} \PY{n+nn}{sklearn}\PY{n+nn}{.}\PY{n+nn}{ensemble}\PY{n+nn}{.}\PY{n+nn}{forest} \PY{k}{import} \PY{n}{RandomForestRegressor}
        \PY{k+kn}{from} \PY{n+nn}{sklearn}\PY{n+nn}{.}\PY{n+nn}{svm} \PY{k}{import} \PY{n}{SVR}
        \PY{k+kn}{from} \PY{n+nn}{sklearn}\PY{n+nn}{.}\PY{n+nn}{metrics} \PY{k}{import} \PY{n}{r2\PYZus{}score}\PY{p}{,} \PY{n}{mean\PYZus{}squared\PYZus{}error}
\end{Verbatim}


    Primeiramente é nescessário conhecer o banco de dados que será utilizado
para obter as respostas das perguntas do teste. Para qualquer banco de
dados é preciso destinguir os tipos de variáveis. Em geral as variáveis
se dividem em qualitativas (categóricas) e quantitaivas, por sua vez
variáveis qualitativas se dividem nominais (por exemplo, sexo) e
ordinais (por exemplo, escolaridade), já as quantitativas se dividem em
discretas e contínuas. Também é importante verificar se todas as
variáveis trazem informações que explicam o fenômeno estudado, pois em
geral, esses problemas envolvem grandes quantidade de dados que exigem
alto custo computacional. Em alguns casos certas variáveis podem ser
combinadas gerando outras que possuem maiores informações.

    \begin{Verbatim}[commandchars=\\\{\}]
{\color{incolor}In [{\color{incolor}2}]:} \PY{c+c1}{\PYZsh{}Importando o banco de dados}
        \PY{n}{dados} \PY{o}{=} \PY{n}{pd}\PY{o}{.}\PY{n}{read\PYZus{}csv}\PY{p}{(}\PY{l+s+s1}{\PYZsq{}}\PY{l+s+s1}{/home/wesley/MEGAsync/teste\PYZus{}estatistico/titanic\PYZus{}orig.csv}\PY{l+s+s1}{\PYZsq{}}\PY{p}{)}
        \PY{n}{dados}\PY{o}{.}\PY{n}{head}\PY{p}{(}\PY{p}{)} \PY{c+c1}{\PYZsh{}Analisando as caracteristicas dos dados, 5 primeiras linhas }
\end{Verbatim}


\begin{Verbatim}[commandchars=\\\{\}]
{\color{outcolor}Out[{\color{outcolor}2}]:}    PassengerId  Survived  Pclass  \textbackslash{}
        0            1         0       3   
        1            2         1       1   
        2            3         1       3   
        3            4         1       1   
        4            5         0       3   
        
                                                        Name     Sex   Age  SibSp  \textbackslash{}
        0                            Braund, Mr. Owen Harris    male  22.0      1   
        1  Cumings, Mrs. John Bradley (Florence Briggs Th{\ldots}  female  38.0      1   
        2                             Heikkinen, Miss. Laina  female  26.0      0   
        3       Futrelle, Mrs. Jacques Heath (Lily May Peel)  female  35.0      1   
        4                           Allen, Mr. William Henry    male  35.0      0   
        
           Parch            Ticket     Fare Cabin Embarked  
        0      0         A/5 21171   7.2500   NaN        S  
        1      0          PC 17599  71.2833   C85        C  
        2      0  STON/O2. 3101282   7.9250   NaN        S  
        3      0            113803  53.1000  C123        S  
        4      0            373450   8.0500   NaN        S  
\end{Verbatim}
            
    Neste banco de dados temos variáveis categóricas e quantitativas
(discretas e contínuas). As variáveis categóricas aprecem na forma
numérica (int) e de texto (str), em python temos que transformar as
variáveis de texto em númericas. Também é possível verificar a
existência de valores faltantes 'NaN' nas variáveis Age, Cabin e
Embarked.

\subsubsection{Valores faltantes}\label{valores-faltantes}

O tratamento de valores faltantes é parte crucial para qualquer análise
de dados, em geral existem diversas formas de tratá-los. A forma mais
simples é substituir NaN's pela média, mediana, moda ou simplesmente (em
casos extremos ou que não gerem perda de informações) a exclusão da
instância. quando se exige maior precisão pode-se optar por abordagens
mais complexas, com uso de modelos de regressão simples ou multivariados
ou de machine learning para estimativa de valores faltantes. Antes de
começar a preencher NaN's é importante verificar se essa variável é
importante para o estudo e quais outras variáveis podem ser utilizadas
para estimar um valor faltante. Existem diferentes formas de imputação
de valores faltantes, algumas são:

Imputação simples

Existindo poucos valores NaN's na variável estudada podemos substituí-lo
utilizando alguma medida de tendência central, é uma forma mais simples
e rápida.

Imputação simples com regressão

Neste método são considerados os valores das demais características para
estimar o valor faltante. Isto pode ser feito baseado em modelos de
regressão logístico (binária), multinomial ou linear.

Imputação múltipla

A Imputação Múltipla é uma técnica para analisar bancos de dados onde
algumas entradas são faltantes. A aplicação dessa técnica requer três
passos: imputação, análise e agrupamento.

O banco estudado possui valores faltantes nas variáveis Age, Cabin e
Embarked. Calculando o total de NaN podemos identificar a melhor forma
de preencher esses valores. Assim, calculamos a soma de valores
faltantes para cada variável:

    \begin{Verbatim}[commandchars=\\\{\}]
{\color{incolor}In [{\color{incolor}3}]:} \PY{n+nb}{print}\PY{p}{(}\PY{l+s+s1}{\PYZsq{}}\PY{l+s+s1}{Soma de valores faltantes em Age:      }\PY{l+s+s1}{\PYZsq{}}\PY{p}{,} \PY{n+nb}{sum}\PY{p}{(}\PY{n}{dados}\PY{p}{[}\PY{l+s+s1}{\PYZsq{}}\PY{l+s+s1}{Age}\PY{l+s+s1}{\PYZsq{}}\PY{p}{]}\PY{o}{.}\PY{n}{isnull}\PY{p}{(}\PY{p}{)}\PY{p}{)}\PY{p}{)}
        \PY{n+nb}{print}\PY{p}{(}\PY{l+s+s1}{\PYZsq{}}\PY{l+s+s1}{Soma de valores faltantes em Cabin:    }\PY{l+s+s1}{\PYZsq{}}\PY{p}{,} \PY{n+nb}{sum}\PY{p}{(}\PY{n}{dados}\PY{p}{[}\PY{l+s+s1}{\PYZsq{}}\PY{l+s+s1}{Cabin}\PY{l+s+s1}{\PYZsq{}}\PY{p}{]}\PY{o}{.}\PY{n}{isnull}\PY{p}{(}\PY{p}{)}\PY{p}{)}\PY{p}{)}
        \PY{n+nb}{print}\PY{p}{(}\PY{l+s+s1}{\PYZsq{}}\PY{l+s+s1}{Soma de valores faltantes em Embarked: }\PY{l+s+s1}{\PYZsq{}}\PY{p}{,} \PY{n+nb}{sum}\PY{p}{(}\PY{n}{dados}\PY{p}{[}\PY{l+s+s1}{\PYZsq{}}\PY{l+s+s1}{Embarked}\PY{l+s+s1}{\PYZsq{}}\PY{p}{]}\PY{o}{.}\PY{n}{isnull}\PY{p}{(}\PY{p}{)}\PY{p}{)}\PY{p}{)}
\end{Verbatim}


    \begin{Verbatim}[commandchars=\\\{\}]
Soma de valores faltantes em Age:       177
Soma de valores faltantes em Cabin:     687
Soma de valores faltantes em Embarked:  2

    \end{Verbatim}

    A variável Cabin é composta de mais de 50\% de valores faltantes.
Estimar esses valores pode trazer um viés para futuras análies, dessa
forma a melhor opção é excluir essa variável de nosso banco de dados.

    \begin{Verbatim}[commandchars=\\\{\}]
{\color{incolor}In [{\color{incolor}4}]:} \PY{n}{dados2} \PY{o}{=} \PY{n}{dados}\PY{o}{.}\PY{n}{drop}\PY{p}{(}\PY{l+s+s1}{\PYZsq{}}\PY{l+s+s1}{Cabin}\PY{l+s+s1}{\PYZsq{}}\PY{p}{,} \PY{n}{axis} \PY{o}{=} \PY{l+m+mi}{1}\PY{p}{)} \PY{c+c1}{\PYZsh{}Excluindo a coluna Cabin}
        \PY{n}{dados2}\PY{o}{.}\PY{n}{head}\PY{p}{(}\PY{p}{)}
\end{Verbatim}


\begin{Verbatim}[commandchars=\\\{\}]
{\color{outcolor}Out[{\color{outcolor}4}]:}    PassengerId  Survived  Pclass  \textbackslash{}
        0            1         0       3   
        1            2         1       1   
        2            3         1       3   
        3            4         1       1   
        4            5         0       3   
        
                                                        Name     Sex   Age  SibSp  \textbackslash{}
        0                            Braund, Mr. Owen Harris    male  22.0      1   
        1  Cumings, Mrs. John Bradley (Florence Briggs Th{\ldots}  female  38.0      1   
        2                             Heikkinen, Miss. Laina  female  26.0      0   
        3       Futrelle, Mrs. Jacques Heath (Lily May Peel)  female  35.0      1   
        4                           Allen, Mr. William Henry    male  35.0      0   
        
           Parch            Ticket     Fare Embarked  
        0      0         A/5 21171   7.2500        S  
        1      0          PC 17599  71.2833        C  
        2      0  STON/O2. 3101282   7.9250        S  
        3      0            113803  53.1000        S  
        4      0            373450   8.0500        S  
\end{Verbatim}
            
    A variável Embarked possui dois valores faltantes, uma abordagem simples
que pode ser utilizada e que não cause viés nas inferências posteriores
é substituir esses valores pela moda, já que estamos trabalhando com
variáveis categóricas. Dessa forma, vamos encontrar a moda e substituir
nos valores falatantes:

    \begin{Verbatim}[commandchars=\\\{\}]
{\color{incolor}In [{\color{incolor}5}]:} \PY{n}{moda} \PY{o}{=} \PY{n}{dados2}\PY{p}{[}\PY{l+s+s1}{\PYZsq{}}\PY{l+s+s1}{Embarked}\PY{l+s+s1}{\PYZsq{}}\PY{p}{]}\PY{o}{.}\PY{n}{mode}\PY{p}{(}\PY{p}{)}
        \PY{n}{dados2}\PY{p}{[}\PY{l+s+s1}{\PYZsq{}}\PY{l+s+s1}{Embarked}\PY{l+s+s1}{\PYZsq{}}\PY{p}{]}\PY{o}{.}\PY{n}{fillna}\PY{p}{(}\PY{n}{moda}\PY{p}{[}\PY{l+m+mi}{0}\PY{p}{]}\PY{p}{,} \PY{n}{inplace} \PY{o}{=} \PY{k+kc}{True}\PY{p}{)}
        \PY{n+nb}{print}\PY{p}{(}\PY{n+nb}{sum}\PY{p}{(}\PY{n}{dados2}\PY{p}{[}\PY{l+s+s1}{\PYZsq{}}\PY{l+s+s1}{Embarked}\PY{l+s+s1}{\PYZsq{}}\PY{p}{]}\PY{o}{.}\PY{n}{isnull}\PY{p}{(}\PY{p}{)}\PY{p}{)}\PY{p}{)}
\end{Verbatim}


    \begin{Verbatim}[commandchars=\\\{\}]
0

    \end{Verbatim}

    A variável idade apresenta 177 valores omissos. A idade é uma variável
mais complexa para ser estimada e sabemos que utilizar a média ou
mediana não é a melhor solução, pois, podemos gerar um viés negativo ou
positivo na nossa análise. A média desses 177 valores aumentarão a
frequência da classe média da idade. Uma forma mais coerente é estimar
as idades faltantes com modelos de regressão simples ou múltiplos ou de
machine learning. Para isso, devemos utilizar o maior número de
variáveis possíveis e verificar se as mesmas contribuem para a
estimativa. Antes de estimar a idade é preciso tentar extrair mais
informações dos dados, pricipalmente verificar se é possível criar novas
variáveis ou excluir as menos importantes.

\subsubsection{Novas variáveis}\label{novas-variuxe1veis}

A combinação de uma ou mais variáveis pode ajudar na estimativa de
sobreviventes do Titanic. Uma variável que pode ser facilmente estimada
é o tamanho da família, pois temos as variáveis SibSp que indica o
número de irmãos ou esposa/esposo e Parch que indica o número de pais e
ou filhos. Somando estas variáveis e acrescentando 1 "a própria pessoa",
temos o tamanho da família:

    \begin{Verbatim}[commandchars=\\\{\}]
{\color{incolor}In [{\color{incolor}6}]:} \PY{c+c1}{\PYZsh{}Tamanho da familia}
        \PY{n}{dados2}\PY{p}{[}\PY{l+s+s1}{\PYZsq{}}\PY{l+s+s1}{Family}\PY{l+s+s1}{\PYZsq{}}\PY{p}{]} \PY{o}{=} \PY{n}{dados2}\PY{p}{[}\PY{l+s+s1}{\PYZsq{}}\PY{l+s+s1}{SibSp}\PY{l+s+s1}{\PYZsq{}}\PY{p}{]} \PY{o}{+} \PY{n}{dados2}\PY{p}{[}\PY{l+s+s1}{\PYZsq{}}\PY{l+s+s1}{Parch}\PY{l+s+s1}{\PYZsq{}}\PY{p}{]} \PY{o}{+} \PY{l+m+mi}{1}
        \PY{n}{dados2}\PY{o}{.}\PY{n}{head}\PY{p}{(}\PY{p}{)}
\end{Verbatim}


\begin{Verbatim}[commandchars=\\\{\}]
{\color{outcolor}Out[{\color{outcolor}6}]:}    PassengerId  Survived  Pclass  \textbackslash{}
        0            1         0       3   
        1            2         1       1   
        2            3         1       3   
        3            4         1       1   
        4            5         0       3   
        
                                                        Name     Sex   Age  SibSp  \textbackslash{}
        0                            Braund, Mr. Owen Harris    male  22.0      1   
        1  Cumings, Mrs. John Bradley (Florence Briggs Th{\ldots}  female  38.0      1   
        2                             Heikkinen, Miss. Laina  female  26.0      0   
        3       Futrelle, Mrs. Jacques Heath (Lily May Peel)  female  35.0      1   
        4                           Allen, Mr. William Henry    male  35.0      0   
        
           Parch            Ticket     Fare Embarked  Family  
        0      0         A/5 21171   7.2500        S       2  
        1      0          PC 17599  71.2833        C       2  
        2      0  STON/O2. 3101282   7.9250        S       1  
        3      0            113803  53.1000        S       2  
        4      0            373450   8.0500        S       1  
\end{Verbatim}
            
    Uma observação importante é, deve-se ter cuidado ao se criar novas
variáveis, pois elas podem ser autocorrelacionadas e, dessa forma,
estariamos inserindo autocorrelção ao nosso modelo. Outro ponto
importante é que nem sempre muitas variáveis explicam melhor os dados, e
o fundamento da regressão é utilizar sempre modelos mais simples para
estimativas.

A variável Name por ser categórica e cada nome representar uma
categoria, não traz muita informação para estimar os sobreviventes.
Entretanto, ela traz um pronome de tratamento ou grau de título, dessa
forma, é importante extrair essa informação e gerar uma nova variável
categórica com o título de cada pessoa.

    \begin{Verbatim}[commandchars=\\\{\}]
{\color{incolor}In [{\color{incolor}7}]:} \PY{c+c1}{\PYZsh{}Quebra a string nome e extrai a informacao do titulo de cada pessoa}
        \PY{n}{dados2}\PY{p}{[}\PY{l+s+s1}{\PYZsq{}}\PY{l+s+s1}{Title}\PY{l+s+s1}{\PYZsq{}}\PY{p}{]} \PY{o}{=} \PY{n}{dados2}\PY{o}{.}\PY{n}{Name}\PY{o}{.}\PY{n}{str}\PY{o}{.}\PY{n}{extract}\PY{p}{(}\PY{l+s+s1}{\PYZsq{}}\PY{l+s+s1}{([A\PYZhy{}Za\PYZhy{}z]+)}\PY{l+s+s1}{\PYZbs{}}\PY{l+s+s1}{.}\PY{l+s+s1}{\PYZsq{}}\PY{p}{,} \PY{n}{expand}\PY{o}{=}\PY{k+kc}{False}\PY{p}{)}
        \PY{n}{dados2}\PY{o}{.}\PY{n}{head}\PY{p}{(}\PY{p}{)}
\end{Verbatim}


\begin{Verbatim}[commandchars=\\\{\}]
{\color{outcolor}Out[{\color{outcolor}7}]:}    PassengerId  Survived  Pclass  \textbackslash{}
        0            1         0       3   
        1            2         1       1   
        2            3         1       3   
        3            4         1       1   
        4            5         0       3   
        
                                                        Name     Sex   Age  SibSp  \textbackslash{}
        0                            Braund, Mr. Owen Harris    male  22.0      1   
        1  Cumings, Mrs. John Bradley (Florence Briggs Th{\ldots}  female  38.0      1   
        2                             Heikkinen, Miss. Laina  female  26.0      0   
        3       Futrelle, Mrs. Jacques Heath (Lily May Peel)  female  35.0      1   
        4                           Allen, Mr. William Henry    male  35.0      0   
        
           Parch            Ticket     Fare Embarked  Family Title  
        0      0         A/5 21171   7.2500        S       2    Mr  
        1      0          PC 17599  71.2833        C       2   Mrs  
        2      0  STON/O2. 3101282   7.9250        S       1  Miss  
        3      0            113803  53.1000        S       2   Mrs  
        4      0            373450   8.0500        S       1    Mr  
\end{Verbatim}
            
    Agora temos os títulos de cada pessoa, entretanto ficamos com 17 títulos
diferentes. Alguns desses títulos são semelhantes ou estão na mesma
classe de grau. Dessa forma, podemos reduzir o número de títulos
agrupando os semelhantes na mesma categoria.

Os títulos dados a homens nobres ou militares com algum estatuto são
Capt, Col, Don, Major, Jonkheer e Sir, já mulheres nobres e de estatuto
social elevado recebem títulos de Dona, Lady e Countess. Os títulos de
Miss e Mlle são para mulheres solteiras, Ms designa uma mulher sem
indicaçao de estado civil, porém nos 2 casos ocorridos, as mesmas
viajavam sozinhas, assim foram consideradas solteiras. As mulheres
casadas são chamadas de Mrs e Mme. Dessa forma, podemos agrupar alguns
desses títulos com outros de maior frequência. O títulos de menor
frequência e atribuídos a pessoas nobres foram substituídos por Rich.

    \begin{Verbatim}[commandchars=\\\{\}]
{\color{incolor}In [{\color{incolor}8}]:} \PY{c+c1}{\PYZsh{}Frequência de titulos}
        \PY{n}{pd}\PY{o}{.}\PY{n}{crosstab}\PY{p}{(}\PY{n}{dados2}\PY{p}{[}\PY{l+s+s1}{\PYZsq{}}\PY{l+s+s1}{Title}\PY{l+s+s1}{\PYZsq{}}\PY{p}{]}\PY{p}{,} \PY{n}{dados2}\PY{p}{[}\PY{l+s+s1}{\PYZsq{}}\PY{l+s+s1}{Sex}\PY{l+s+s1}{\PYZsq{}}\PY{p}{]}\PY{p}{)}
\end{Verbatim}


\begin{Verbatim}[commandchars=\\\{\}]
{\color{outcolor}Out[{\color{outcolor}8}]:} Sex       female  male
        Title                 
        Capt           0     1
        Col            0     2
        Countess       1     0
        Don            0     1
        Dr             1     6
        Jonkheer       0     1
        Lady           1     0
        Major          0     2
        Master         0    40
        Miss         182     0
        Mlle           2     0
        Mme            1     0
        Mr             0   517
        Mrs          125     0
        Ms             1     0
        Rev            0     6
        Sir            0     1
\end{Verbatim}
            
    \begin{Verbatim}[commandchars=\\\{\}]
{\color{incolor}In [{\color{incolor}9}]:} \PY{c+c1}{\PYZsh{}Agrupando passageiros em titulos mais comuns}
        \PY{n}{dados2}\PY{p}{[}\PY{l+s+s1}{\PYZsq{}}\PY{l+s+s1}{Title}\PY{l+s+s1}{\PYZsq{}}\PY{p}{]} \PY{o}{=} \PY{n}{dados2}\PY{p}{[}\PY{l+s+s1}{\PYZsq{}}\PY{l+s+s1}{Title}\PY{l+s+s1}{\PYZsq{}}\PY{p}{]}\PY{o}{.}\PY{n}{replace}\PY{p}{(}\PY{p}{[}\PY{l+s+s1}{\PYZsq{}}\PY{l+s+s1}{Jonkheer}\PY{l+s+s1}{\PYZsq{}}\PY{p}{,} \PY{l+s+s1}{\PYZsq{}}\PY{l+s+s1}{Don}\PY{l+s+s1}{\PYZsq{}}\PY{p}{,} \PY{l+s+s1}{\PYZsq{}}\PY{l+s+s1}{Capt}\PY{l+s+s1}{\PYZsq{}}\PY{p}{,} \PY{l+s+s1}{\PYZsq{}}\PY{l+s+s1}{Major}\PY{l+s+s1}{\PYZsq{}}\PY{p}{,} \PY{l+s+s1}{\PYZsq{}}\PY{l+s+s1}{Col}\PY{l+s+s1}{\PYZsq{}}\PY{p}{,} \PY{l+s+s1}{\PYZsq{}}\PY{l+s+s1}{Countess}\PY{l+s+s1}{\PYZsq{}}\PY{p}{,} \PY{l+s+s1}{\PYZsq{}}\PY{l+s+s1}{Lady}\PY{l+s+s1}{\PYZsq{}}\PY{p}{,}
                                                  \PY{l+s+s1}{\PYZsq{}}\PY{l+s+s1}{Dr}\PY{l+s+s1}{\PYZsq{}}\PY{p}{,} \PY{l+s+s1}{\PYZsq{}}\PY{l+s+s1}{Rev}\PY{l+s+s1}{\PYZsq{}}\PY{p}{,} \PY{l+s+s1}{\PYZsq{}}\PY{l+s+s1}{Sir}\PY{l+s+s1}{\PYZsq{}}\PY{p}{]}\PY{p}{,} \PY{l+s+s1}{\PYZsq{}}\PY{l+s+s1}{Rich}\PY{l+s+s1}{\PYZsq{}}\PY{p}{)}
        \PY{n}{dados2}\PY{p}{[}\PY{l+s+s1}{\PYZsq{}}\PY{l+s+s1}{Title}\PY{l+s+s1}{\PYZsq{}}\PY{p}{]} \PY{o}{=} \PY{n}{dados2}\PY{p}{[}\PY{l+s+s1}{\PYZsq{}}\PY{l+s+s1}{Title}\PY{l+s+s1}{\PYZsq{}}\PY{p}{]}\PY{o}{.}\PY{n}{replace}\PY{p}{(}\PY{p}{[}\PY{l+s+s1}{\PYZsq{}}\PY{l+s+s1}{Mlle}\PY{l+s+s1}{\PYZsq{}}\PY{p}{,} \PY{l+s+s1}{\PYZsq{}}\PY{l+s+s1}{Ms}\PY{l+s+s1}{\PYZsq{}}\PY{p}{]}\PY{p}{,} \PY{l+s+s1}{\PYZsq{}}\PY{l+s+s1}{Miss}\PY{l+s+s1}{\PYZsq{}}\PY{p}{)}
        \PY{n}{dados2}\PY{p}{[}\PY{l+s+s1}{\PYZsq{}}\PY{l+s+s1}{Title}\PY{l+s+s1}{\PYZsq{}}\PY{p}{]} \PY{o}{=} \PY{n}{dados2}\PY{p}{[}\PY{l+s+s1}{\PYZsq{}}\PY{l+s+s1}{Title}\PY{l+s+s1}{\PYZsq{}}\PY{p}{]}\PY{o}{.}\PY{n}{replace}\PY{p}{(}\PY{p}{[}\PY{l+s+s1}{\PYZsq{}}\PY{l+s+s1}{Mme}\PY{l+s+s1}{\PYZsq{}}\PY{p}{]}\PY{p}{,} \PY{l+s+s1}{\PYZsq{}}\PY{l+s+s1}{Mrs}\PY{l+s+s1}{\PYZsq{}}\PY{p}{)}
\end{Verbatim}


    \begin{Verbatim}[commandchars=\\\{\}]
{\color{incolor}In [{\color{incolor}10}]:} \PY{c+c1}{\PYZsh{}Titulos por sexo}
         \PY{n}{pd}\PY{o}{.}\PY{n}{crosstab}\PY{p}{(}\PY{n}{dados2}\PY{p}{[}\PY{l+s+s1}{\PYZsq{}}\PY{l+s+s1}{Title}\PY{l+s+s1}{\PYZsq{}}\PY{p}{]}\PY{p}{,} \PY{n}{dados2}\PY{p}{[}\PY{l+s+s1}{\PYZsq{}}\PY{l+s+s1}{Sex}\PY{l+s+s1}{\PYZsq{}}\PY{p}{]}\PY{p}{)}
\end{Verbatim}


\begin{Verbatim}[commandchars=\\\{\}]
{\color{outcolor}Out[{\color{outcolor}10}]:} Sex     female  male
         Title               
         Master       0    40
         Miss       185     0
         Mr           0   517
         Mrs        126     0
         Rich         3    20
\end{Verbatim}
            
    Depois desse agrupamento ficamos com 5 categorias na variável título.
Agora a variável Name não é mais importante podendo ser excluída do
nosso banco de dados. Outra variável que ainda não foi tratada é a
Ticket. Analisando um pouco essa variável pode-se notar que ela
apresenta valores numéricos e de texto, em geral as letras podem indicar
uma classe de valores dos tickets. Porém, não existe uma descrição para
essa variável e as informações obtidas pela Classe social e pelo valor
pago da passagem trazem bastante informação sobre as condições dos
passageiros. Assim, essas variáveis podem ser excluídas do banco de
dados.

    \begin{Verbatim}[commandchars=\\\{\}]
{\color{incolor}In [{\color{incolor}11}]:} \PY{c+c1}{\PYZsh{}Excluindo Name e Ticket}
         \PY{n}{dados3} \PY{o}{=} \PY{n}{dados2}\PY{o}{.}\PY{n}{drop}\PY{p}{(}\PY{p}{[}\PY{l+s+s1}{\PYZsq{}}\PY{l+s+s1}{Name}\PY{l+s+s1}{\PYZsq{}}\PY{p}{,} \PY{l+s+s1}{\PYZsq{}}\PY{l+s+s1}{Ticket}\PY{l+s+s1}{\PYZsq{}}\PY{p}{]}\PY{p}{,} \PY{n}{axis} \PY{o}{=} \PY{l+m+mi}{1}\PY{p}{)}
         \PY{n}{dados3}\PY{o}{.}\PY{n}{head}\PY{p}{(}\PY{p}{)}
\end{Verbatim}


\begin{Verbatim}[commandchars=\\\{\}]
{\color{outcolor}Out[{\color{outcolor}11}]:}    PassengerId  Survived  Pclass     Sex   Age  SibSp  Parch     Fare  \textbackslash{}
         0            1         0       3    male  22.0      1      0   7.2500   
         1            2         1       1  female  38.0      1      0  71.2833   
         2            3         1       3  female  26.0      0      0   7.9250   
         3            4         1       1  female  35.0      1      0  53.1000   
         4            5         0       3    male  35.0      0      0   8.0500   
         
           Embarked  Family Title  
         0        S       2    Mr  
         1        C       2   Mrs  
         2        S       1  Miss  
         3        S       2   Mrs  
         4        S       1    Mr  
\end{Verbatim}
            
    \subsubsection{Variáveis categóricas e
contínuas}\label{variuxe1veis-categuxf3ricas-e-contuxednuas}

    O banco de dados está quase pronto para podermos realizar nossa
exploração. Sabemos que nosso banco é composto de variáveis
quantitativas (contínuas e discretas) e qualitativas "ou categóricas"
(nominal e ordinal). As variáveis quantitativas estão prontas para
análise, porém as categóricas precisam ser ajustadas para trabalharmos
com python. A melhor forma de trabalhar com essas variáveis é atribuir
valores para cada atributo, com isso não teremos problemas posteriores.

    \begin{Verbatim}[commandchars=\\\{\}]
{\color{incolor}In [{\color{incolor}12}]:} \PY{c+c1}{\PYZsh{}modulo LabelEncoder da biblioteca sklearn}
         \PY{n}{labelencoder} \PY{o}{=} \PY{n}{LabelEncoder}\PY{p}{(}\PY{p}{)}
         
         \PY{n}{dados4} \PY{o}{=} \PY{n}{dados3}
         \PY{n}{dados4}\PY{p}{[}\PY{l+s+s1}{\PYZsq{}}\PY{l+s+s1}{Sex}\PY{l+s+s1}{\PYZsq{}}\PY{p}{]} \PY{o}{=} \PY{n}{labelencoder}\PY{o}{.}\PY{n}{fit\PYZus{}transform}\PY{p}{(}\PY{n}{dados4}\PY{p}{[}\PY{l+s+s1}{\PYZsq{}}\PY{l+s+s1}{Sex}\PY{l+s+s1}{\PYZsq{}}\PY{p}{]}\PY{p}{)}
         \PY{n}{dados4}\PY{p}{[}\PY{l+s+s1}{\PYZsq{}}\PY{l+s+s1}{Embarked}\PY{l+s+s1}{\PYZsq{}}\PY{p}{]} \PY{o}{=} \PY{n}{labelencoder}\PY{o}{.}\PY{n}{fit\PYZus{}transform}\PY{p}{(}\PY{n}{dados4}\PY{p}{[}\PY{l+s+s1}{\PYZsq{}}\PY{l+s+s1}{Embarked}\PY{l+s+s1}{\PYZsq{}}\PY{p}{]}\PY{p}{)}
         \PY{n}{dados4}\PY{p}{[}\PY{l+s+s1}{\PYZsq{}}\PY{l+s+s1}{Title}\PY{l+s+s1}{\PYZsq{}}\PY{p}{]} \PY{o}{=} \PY{n}{labelencoder}\PY{o}{.}\PY{n}{fit\PYZus{}transform}\PY{p}{(}\PY{n}{dados4}\PY{p}{[}\PY{l+s+s1}{\PYZsq{}}\PY{l+s+s1}{Title}\PY{l+s+s1}{\PYZsq{}}\PY{p}{]}\PY{p}{)}
         \PY{n}{dados4}\PY{o}{.}\PY{n}{head}\PY{p}{(}\PY{p}{)}
         
         \PY{c+c1}{\PYZsh{}Codificacoes}
         \PY{c+c1}{\PYZsh{}Sex}
         \PY{c+c1}{\PYZsh{}female = 0, male = 1,}
         \PY{c+c1}{\PYZsh{}Embarked}
         \PY{c+c1}{\PYZsh{}C = 0, Q = 1, S = 2}
         \PY{c+c1}{\PYZsh{}Title}
         \PY{c+c1}{\PYZsh{}Master = 0, Miss = 1, Mr = 2, Mrs = 3, Rich = 4}
\end{Verbatim}


\begin{Verbatim}[commandchars=\\\{\}]
{\color{outcolor}Out[{\color{outcolor}12}]:}    PassengerId  Survived  Pclass  Sex   Age  SibSp  Parch     Fare  Embarked  \textbackslash{}
         0            1         0       3    1  22.0      1      0   7.2500         2   
         1            2         1       1    0  38.0      1      0  71.2833         0   
         2            3         1       3    0  26.0      0      0   7.9250         2   
         3            4         1       1    0  35.0      1      0  53.1000         2   
         4            5         0       3    1  35.0      0      0   8.0500         2   
         
            Family  Title  
         0       2      2  
         1       2      3  
         2       1      1  
         3       2      3  
         4       1      2  
\end{Verbatim}
            
    \subsubsection{Preenchendo NaN's de
Idade}\label{preenchendo-nans-de-idade}

    Com o banco de dados pronto é possível estimar os valores para idade.
Seguiremos duas abordagens para identificarmos qual a mais apropriada. A
primeira abordagem é ajustar um modelo linear múltiplo utilizando as
demais variáveis para estimar a idade, a segunda abordagem é utilizar
modelos de machine learning, Randon Forest e Máquinas de Vetores de
Suporte. O modelo que obtiver melhores scores será utlizado para estimar
a idade.

    \begin{Verbatim}[commandchars=\\\{\}]
{\color{incolor}In [{\color{incolor}13}]:} \PY{c+c1}{\PYZsh{}Gerando um banco de dados para idade}
         
         \PY{n}{dados\PYZus{}idade} \PY{o}{=} \PY{n}{dados4}\PY{p}{[}\PY{p}{[}\PY{l+s+s1}{\PYZsq{}}\PY{l+s+s1}{Age}\PY{l+s+s1}{\PYZsq{}}\PY{p}{,} \PY{l+s+s1}{\PYZsq{}}\PY{l+s+s1}{Pclass}\PY{l+s+s1}{\PYZsq{}}\PY{p}{,} \PY{l+s+s1}{\PYZsq{}}\PY{l+s+s1}{Title}\PY{l+s+s1}{\PYZsq{}}\PY{p}{,} \PY{l+s+s1}{\PYZsq{}}\PY{l+s+s1}{Embarked}\PY{l+s+s1}{\PYZsq{}}\PY{p}{,} \PY{l+s+s1}{\PYZsq{}}\PY{l+s+s1}{Sex}\PY{l+s+s1}{\PYZsq{}}\PY{p}{,} \PY{l+s+s1}{\PYZsq{}}\PY{l+s+s1}{SibSp}\PY{l+s+s1}{\PYZsq{}}\PY{p}{,} \PY{l+s+s1}{\PYZsq{}}\PY{l+s+s1}{Parch}\PY{l+s+s1}{\PYZsq{}}\PY{p}{,} \PY{l+s+s1}{\PYZsq{}}\PY{l+s+s1}{Fare}\PY{l+s+s1}{\PYZsq{}}\PY{p}{,} \PY{l+s+s1}{\PYZsq{}}\PY{l+s+s1}{Family}\PY{l+s+s1}{\PYZsq{}} \PY{p}{]}\PY{p}{]}
         
         \PY{c+c1}{\PYZsh{}Queremos estimar a idade para isso vamos usar apenas as intâncias com as idades completas}
         \PY{c+c1}{\PYZsh{}Divindo o banco em idade\PYZus{}conhecida e idade\PYZus{}desconhecida}
         
         \PY{n}{idade\PYZus{}conh} \PY{o}{=} \PY{n}{dados\PYZus{}idade}\PY{o}{.}\PY{n}{loc}\PY{p}{[}\PY{p}{(}\PY{n}{dados\PYZus{}idade}\PY{o}{.}\PY{n}{Age}\PY{o}{.}\PY{n}{notnull}\PY{p}{(}\PY{p}{)}\PY{p}{)}\PY{p}{]}
         \PY{n}{idade\PYZus{}desc} \PY{o}{=} \PY{n}{dados\PYZus{}idade}\PY{o}{.}\PY{n}{loc}\PY{p}{[}\PY{p}{(}\PY{n}{dados\PYZus{}idade}\PY{o}{.}\PY{n}{Age}\PY{o}{.}\PY{n}{isnull}\PY{p}{(}\PY{p}{)}\PY{p}{)}\PY{p}{]}
         
         \PY{c+c1}{\PYZsh{}Definindo a variável resposta (dependente) e as variáveis preditoras (independentes)}
         
         \PY{n}{y} \PY{o}{=} \PY{n}{idade\PYZus{}conh}\PY{o}{.}\PY{n}{values}\PY{p}{[}\PY{p}{:}\PY{p}{,} \PY{l+m+mi}{0}\PY{p}{]}
         \PY{n}{X} \PY{o}{=} \PY{n}{idade\PYZus{}conh}\PY{o}{.}\PY{n}{values}\PY{p}{[}\PY{p}{:}\PY{p}{,} \PY{l+m+mi}{1}\PY{p}{:}\PY{p}{:}\PY{p}{]}
         
         \PY{c+c1}{\PYZsh{}Preparando as variáveis para estimar os valor de idade perdidas}
         \PY{n}{X\PYZus{}p} \PY{o}{=} \PY{n}{idade\PYZus{}desc}\PY{o}{.}\PY{n}{values}\PY{p}{[}\PY{p}{:}\PY{p}{,} \PY{l+m+mi}{1}\PY{p}{:}\PY{p}{:}\PY{p}{]}
\end{Verbatim}


    Na existência de variáveis categóricas, geralmente substituímos as
categorias por números inteiros, porém em modelos de machine learning,
isso pode causar um viés, pois o algoritmo pode considerar que existe
uma relação de ordem na variável. Uma forma de corrigir esse problema é
com uso de variáveis dummy, esse procedimento conciste em criar um vetor
para cada fator e adicionar um valor binário 0 ou 1, para a ocorrência
do fator, logo uma variável dummy com 3 fatores será substituída por 3
vetores. Outra forma de diminuir o ruído dos dados é normalizar as
variáveis, isso diminui a variância dos dados permitindo melhores
ajustes.

    \begin{Verbatim}[commandchars=\\\{\}]
{\color{incolor}In [{\color{incolor}14}]:} \PY{c+c1}{\PYZsh{}Atribuindo variaveis dummy a variaveis categoricas}
         \PY{n}{onehotencoder} \PY{o}{=} \PY{n}{OneHotEncoder}\PY{p}{(}\PY{n}{categorical\PYZus{}features} \PY{o}{=} \PY{p}{[}\PY{l+m+mi}{0}\PY{p}{,} \PY{l+m+mi}{1}\PY{p}{,} \PY{l+m+mi}{2}\PY{p}{]}\PY{p}{)}
         \PY{n}{X\PYZus{}treino} \PY{o}{=} \PY{n}{onehotencoder}\PY{o}{.}\PY{n}{fit\PYZus{}transform}\PY{p}{(}\PY{n}{X}\PY{p}{)}\PY{o}{.}\PY{n}{toarray}\PY{p}{(}\PY{p}{)}
         \PY{n}{X\PYZus{}est} \PY{o}{=} \PY{n}{onehotencoder}\PY{o}{.}\PY{n}{fit\PYZus{}transform}\PY{p}{(}\PY{n}{X\PYZus{}p}\PY{p}{)}\PY{o}{.}\PY{n}{toarray}\PY{p}{(}\PY{p}{)}
         
         \PY{c+c1}{\PYZsh{}Normalizando os dados}
         \PY{n}{sc} \PY{o}{=} \PY{n}{StandardScaler}\PY{p}{(}\PY{p}{)}
         \PY{n}{X\PYZus{}treino} \PY{o}{=} \PY{n}{sc}\PY{o}{.}\PY{n}{fit\PYZus{}transform}\PY{p}{(}\PY{n}{X\PYZus{}treino}\PY{p}{)}
         \PY{n}{X\PYZus{}est} \PY{o}{=} \PY{n}{sc}\PY{o}{.}\PY{n}{transform}\PY{p}{(}\PY{n}{X\PYZus{}est}\PY{p}{)}
\end{Verbatim}


    \begin{Verbatim}[commandchars=\\\{\}]
{\color{incolor}In [{\color{incolor}15}]:} \PY{c+c1}{\PYZsh{}Definindo linear multiplo}
         \PY{n}{modelo\PYZus{}linear} \PY{o}{=} \PY{n}{LinearRegression}\PY{p}{(}\PY{p}{)}
         
         \PY{c+c1}{\PYZsh{}Ajustando o modelo}
         \PY{n}{y\PYZus{}pred\PYZus{}linear} \PY{o}{=} \PY{n}{modelo\PYZus{}linear}\PY{o}{.}\PY{n}{fit}\PY{p}{(}\PY{n}{X\PYZus{}treino}\PY{p}{,} \PY{n}{y}\PY{p}{)}
         \PY{n+nb}{print}\PY{p}{(}\PY{n}{r2\PYZus{}score}\PY{p}{(}\PY{n}{y}\PY{p}{,} \PY{n}{modelo\PYZus{}linear}\PY{o}{.}\PY{n}{predict}\PY{p}{(}\PY{n}{X\PYZus{}treino}\PY{p}{)}\PY{p}{)}\PY{p}{)}    
         \PY{n}{mean\PYZus{}squared\PYZus{}error}\PY{p}{(}\PY{n}{y}\PY{p}{,} \PY{n}{modelo\PYZus{}linear}\PY{o}{.}\PY{n}{predict}\PY{p}{(}\PY{n}{X\PYZus{}treino}\PY{p}{)}\PY{p}{)} \PY{c+c1}{\PYZsh{}Erro quadratico medio}
\end{Verbatim}


    \begin{Verbatim}[commandchars=\\\{\}]
0.424509368962

    \end{Verbatim}

\begin{Verbatim}[commandchars=\\\{\}]
{\color{outcolor}Out[{\color{outcolor}15}]:} 121.26944588696158
\end{Verbatim}
            
    \begin{Verbatim}[commandchars=\\\{\}]
{\color{incolor}In [{\color{incolor}16}]:} \PY{c+c1}{\PYZsh{}Definindo o modelo Random Forest}
         
         \PY{n}{modelo\PYZus{}rf} \PY{o}{=} \PY{n}{RandomForestRegressor}\PY{p}{(}\PY{n}{n\PYZus{}estimators}\PY{o}{=}\PY{l+m+mi}{1000}\PY{p}{)}
         \PY{n}{modelo\PYZus{}rf}\PY{o}{.}\PY{n}{fit}\PY{p}{(}\PY{n}{X\PYZus{}treino}\PY{p}{,} \PY{n}{y}\PY{p}{)}
         \PY{n+nb}{print}\PY{p}{(}\PY{n}{r2\PYZus{}score}\PY{p}{(}\PY{n}{y}\PY{p}{,} \PY{n}{modelo\PYZus{}rf}\PY{o}{.}\PY{n}{predict}\PY{p}{(}\PY{n}{X\PYZus{}treino}\PY{p}{)}\PY{p}{)}\PY{p}{)}
         \PY{n}{mean\PYZus{}squared\PYZus{}error}\PY{p}{(}\PY{n}{y}\PY{p}{,} \PY{n}{modelo\PYZus{}rf}\PY{o}{.}\PY{n}{predict}\PY{p}{(}\PY{n}{X\PYZus{}treino}\PY{p}{)}\PY{p}{)}
\end{Verbatim}


    \begin{Verbatim}[commandchars=\\\{\}]
0.740413566285

    \end{Verbatim}

\begin{Verbatim}[commandchars=\\\{\}]
{\color{outcolor}Out[{\color{outcolor}16}]:} 54.700982567842566
\end{Verbatim}
            
    \begin{Verbatim}[commandchars=\\\{\}]
{\color{incolor}In [{\color{incolor}17}]:} \PY{c+c1}{\PYZsh{}Definindo o modelo Support Verctor Regression}
         
         \PY{n}{modelo\PYZus{}svr} \PY{o}{=} \PY{n}{SVR}\PY{p}{(}\PY{n}{kernel} \PY{o}{=} \PY{l+s+s1}{\PYZsq{}}\PY{l+s+s1}{rbf}\PY{l+s+s1}{\PYZsq{}}\PY{p}{,} \PY{n}{C}\PY{o}{=}\PY{l+m+mf}{1e3}\PY{p}{,} \PY{n}{gamma}\PY{o}{=}\PY{l+m+mf}{0.1}\PY{p}{)}
         \PY{n}{modelo\PYZus{}svr}\PY{o}{.}\PY{n}{fit}\PY{p}{(}\PY{n}{X\PYZus{}treino}\PY{p}{,} \PY{n}{y}\PY{p}{)}
         \PY{n+nb}{print}\PY{p}{(}\PY{n}{r2\PYZus{}score}\PY{p}{(}\PY{n}{y}\PY{p}{,} \PY{n}{modelo\PYZus{}svr}\PY{o}{.}\PY{n}{predict}\PY{p}{(}\PY{n}{X\PYZus{}treino}\PY{p}{)}\PY{p}{)}\PY{p}{)}
         \PY{n}{mean\PYZus{}squared\PYZus{}error}\PY{p}{(}\PY{n}{y}\PY{p}{,} \PY{n}{modelo\PYZus{}svr}\PY{o}{.}\PY{n}{predict}\PY{p}{(}\PY{n}{X\PYZus{}treino}\PY{p}{)}\PY{p}{)}
\end{Verbatim}


    \begin{Verbatim}[commandchars=\\\{\}]
0.56446071797

    \end{Verbatim}

\begin{Verbatim}[commandchars=\\\{\}]
{\color{outcolor}Out[{\color{outcolor}17}]:} 91.778396632773209
\end{Verbatim}
            
    O modelo que apresentou melhores resultados foi o Random Forest. Vamos
preencher o valores faltantes com o resultado do melhor modelo.

    \begin{Verbatim}[commandchars=\\\{\}]
{\color{incolor}In [{\color{incolor}18}]:} \PY{n}{idades\PYZus{}pred} \PY{o}{=} \PY{n}{modelo\PYZus{}rf}\PY{o}{.}\PY{n}{predict}\PY{p}{(}\PY{n}{X\PYZus{}est}\PY{p}{)}
         
         \PY{c+c1}{\PYZsh{}Substituindo no banco de dados}
         \PY{n}{dados4}\PY{o}{.}\PY{n}{loc}\PY{p}{[}\PY{p}{(}\PY{n}{dados4}\PY{o}{.}\PY{n}{Age}\PY{o}{.}\PY{n}{isnull}\PY{p}{(}\PY{p}{)}\PY{p}{)}\PY{p}{,} \PY{l+s+s1}{\PYZsq{}}\PY{l+s+s1}{Age}\PY{l+s+s1}{\PYZsq{}}\PY{p}{]} \PY{o}{=} \PY{n}{idades\PYZus{}pred}
\end{Verbatim}


    Agora vamos verificar se os valores estimados se distribuem próximo dos
valores reais e assim podemos verifar se o modelo é o ideal para
preencher valores faltantes na idade. Podemos visualizar isso pelo
gráfico de histograma.

    \begin{Verbatim}[commandchars=\\\{\}]
{\color{incolor}In [{\color{incolor}19}]:} \PY{n}{plt}\PY{o}{.}\PY{n}{figure}\PY{p}{(}\PY{n}{figsize}\PY{o}{=}\PY{p}{(}\PY{l+m+mi}{10}\PY{p}{,}\PY{l+m+mi}{7}\PY{p}{)}\PY{p}{)}
         \PY{n}{ax} \PY{o}{=} \PY{n}{sns}\PY{o}{.}\PY{n}{kdeplot}\PY{p}{(}\PY{n}{dados4}\PY{p}{[}\PY{l+s+s1}{\PYZsq{}}\PY{l+s+s1}{Age}\PY{l+s+s1}{\PYZsq{}}\PY{p}{]}\PY{p}{,} \PY{n}{shade} \PY{o}{=} \PY{k+kc}{True}\PY{p}{,} \PY{n}{label} \PY{o}{=} \PY{l+s+s1}{\PYZsq{}}\PY{l+s+s1}{Idade Estimada}\PY{l+s+s1}{\PYZsq{}}\PY{p}{)}
         \PY{n}{sns}\PY{o}{.}\PY{n}{kdeplot}\PY{p}{(}\PY{n}{idade\PYZus{}conh}\PY{p}{[}\PY{l+s+s1}{\PYZsq{}}\PY{l+s+s1}{Age}\PY{l+s+s1}{\PYZsq{}}\PY{p}{]}\PY{p}{,} \PY{n}{shade} \PY{o}{=} \PY{k+kc}{True}\PY{p}{,} \PY{n}{label} \PY{o}{=} \PY{l+s+s1}{\PYZsq{}}\PY{l+s+s1}{Idade Real}\PY{l+s+s1}{\PYZsq{}}\PY{p}{)}
         \PY{n}{ax}\PY{o}{.}\PY{n}{set}\PY{p}{(}\PY{n}{xlabel}\PY{o}{=}\PY{l+s+s1}{\PYZsq{}}\PY{l+s+s1}{Idade}\PY{l+s+s1}{\PYZsq{}}\PY{p}{,} \PY{n}{ylabel}\PY{o}{=}\PY{l+s+s1}{\PYZsq{}}\PY{l+s+s1}{Densidade}\PY{l+s+s1}{\PYZsq{}}\PY{p}{)}
         \PY{n}{sns}\PY{o}{.}\PY{n}{plt}\PY{o}{.}\PY{n}{show}\PY{p}{(}\PY{p}{)}
\end{Verbatim}


    \begin{center}
    \adjustimage{max size={0.9\linewidth}{0.9\paperheight}}{output_35_0.png}
    \end{center}
    { \hspace*{\fill} \\}
    
    \subsection{Estatística Descritiva}\label{estatuxedstica-descritiva}

    Como o banco de dados completo, podemos fazer diversas inferências sobre
o incidente com o Titanic. Uma análise descritiva é primordial para nos
dar conhecimento prévio sobre a distribuição dos dados. Nesta análise
utilizaremos medidas de resumo e gráficos, também começaremos a
responder as perguntas do teste.

Primeiro vamos verificar o número de passageiros por sexo e classe
social.

    \begin{Verbatim}[commandchars=\\\{\}]
{\color{incolor}In [{\color{incolor}20}]:} \PY{c+c1}{\PYZsh{}Palette de cores para os graficos}
         \PY{n}{tabela\PYZus{}cores} \PY{o}{=} \PY{p}{[}\PY{l+s+s1}{\PYZsq{}}\PY{l+s+s1}{\PYZsh{}78C850}\PY{l+s+s1}{\PYZsq{}}\PY{p}{,}  \PY{c+c1}{\PYZsh{} Grass}
                          \PY{l+s+s1}{\PYZsq{}}\PY{l+s+s1}{\PYZsh{}F08030}\PY{l+s+s1}{\PYZsq{}}\PY{p}{,}  \PY{c+c1}{\PYZsh{} Fire}
                          \PY{l+s+s1}{\PYZsq{}}\PY{l+s+s1}{\PYZsh{}6890F0}\PY{l+s+s1}{\PYZsq{}}\PY{p}{,}  \PY{c+c1}{\PYZsh{} Water}
                          \PY{l+s+s1}{\PYZsq{}}\PY{l+s+s1}{\PYZsh{}A8B820}\PY{l+s+s1}{\PYZsq{}}\PY{p}{,}  \PY{c+c1}{\PYZsh{} Bug}
                          \PY{l+s+s1}{\PYZsq{}}\PY{l+s+s1}{\PYZsh{}A8A878}\PY{l+s+s1}{\PYZsq{}}\PY{p}{,}  \PY{c+c1}{\PYZsh{} Normal}
                          \PY{l+s+s1}{\PYZsq{}}\PY{l+s+s1}{\PYZsh{}A040A0}\PY{l+s+s1}{\PYZsq{}}\PY{p}{,}  \PY{c+c1}{\PYZsh{} Poison}
                          \PY{l+s+s1}{\PYZsq{}}\PY{l+s+s1}{\PYZsh{}F8D030}\PY{l+s+s1}{\PYZsq{}}\PY{p}{,}  \PY{c+c1}{\PYZsh{} Electric}
                          \PY{l+s+s1}{\PYZsq{}}\PY{l+s+s1}{\PYZsh{}E0C068}\PY{l+s+s1}{\PYZsq{}}\PY{p}{,}  \PY{c+c1}{\PYZsh{} Ground}
                          \PY{l+s+s1}{\PYZsq{}}\PY{l+s+s1}{\PYZsh{}EE99AC}\PY{l+s+s1}{\PYZsq{}}\PY{p}{,}  \PY{c+c1}{\PYZsh{} Fairy}
                          \PY{l+s+s1}{\PYZsq{}}\PY{l+s+s1}{\PYZsh{}C03028}\PY{l+s+s1}{\PYZsq{}}\PY{p}{,}  \PY{c+c1}{\PYZsh{} Fighting}
                          \PY{l+s+s1}{\PYZsq{}}\PY{l+s+s1}{\PYZsh{}F85888}\PY{l+s+s1}{\PYZsq{}}\PY{p}{,}  \PY{c+c1}{\PYZsh{} Psychic}
                          \PY{l+s+s1}{\PYZsq{}}\PY{l+s+s1}{\PYZsh{}B8A038}\PY{l+s+s1}{\PYZsq{}}\PY{p}{,}  \PY{c+c1}{\PYZsh{} Rock}
                          \PY{l+s+s1}{\PYZsq{}}\PY{l+s+s1}{\PYZsh{}705898}\PY{l+s+s1}{\PYZsq{}}\PY{p}{,}  \PY{c+c1}{\PYZsh{} Ghost}
                          \PY{l+s+s1}{\PYZsq{}}\PY{l+s+s1}{\PYZsh{}98D8D8}\PY{l+s+s1}{\PYZsq{}}\PY{p}{,}  \PY{c+c1}{\PYZsh{} Ice}
                          \PY{l+s+s1}{\PYZsq{}}\PY{l+s+s1}{\PYZsh{}7038F8}\PY{l+s+s1}{\PYZsq{}}\PY{p}{,}  \PY{c+c1}{\PYZsh{} Dragon}
                         \PY{p}{]}
\end{Verbatim}


    \begin{Verbatim}[commandchars=\\\{\}]
{\color{incolor}In [{\color{incolor}21}]:} \PY{c+c1}{\PYZsh{}Graficos de numero de passageiros por sexo e classe}
         \PY{n}{plt}\PY{o}{.}\PY{n}{subplots}\PY{p}{(}\PY{n}{figsize}\PY{o}{=}\PY{p}{(}\PY{p}{[}\PY{l+m+mi}{17}\PY{p}{,}\PY{l+m+mi}{6}\PY{p}{]}\PY{p}{)}\PY{p}{)}
         \PY{n}{plt}\PY{o}{.}\PY{n}{subplot}\PY{p}{(}\PY{l+m+mi}{121}\PY{p}{)}
         \PY{n}{ax} \PY{o}{=} \PY{n}{sns}\PY{o}{.}\PY{n}{barplot}\PY{p}{(}\PY{n}{dados4}\PY{p}{[}\PY{l+s+s1}{\PYZsq{}}\PY{l+s+s1}{Sex}\PY{l+s+s1}{\PYZsq{}}\PY{p}{]}\PY{p}{,} \PY{n}{dados4}\PY{o}{.}\PY{n}{Sex}\PY{o}{.}\PY{n}{value\PYZus{}counts}\PY{p}{(}\PY{p}{)}\PY{p}{,} \PY{n}{ci} \PY{o}{=} \PY{k+kc}{None}\PY{p}{,} \PY{n}{palette} \PY{o}{=} \PY{n}{tabela\PYZus{}cores}\PY{p}{)}
         \PY{n}{ax}\PY{o}{.}\PY{n}{set\PYZus{}xticklabels}\PY{p}{(}\PY{n}{labels} \PY{o}{=} \PY{p}{[}\PY{l+s+s1}{\PYZsq{}}\PY{l+s+s1}{Masculino}\PY{l+s+s1}{\PYZsq{}}\PY{p}{,} \PY{l+s+s1}{\PYZsq{}}\PY{l+s+s1}{Feminino}\PY{l+s+s1}{\PYZsq{}}\PY{p}{]}\PY{p}{)}
         \PY{n}{ax}\PY{o}{.}\PY{n}{set}\PY{p}{(}\PY{n}{xlabel}\PY{o}{=}\PY{l+s+s1}{\PYZsq{}}\PY{l+s+s1}{\PYZsq{}}\PY{p}{,} \PY{n}{ylabel}\PY{o}{=}\PY{l+s+s1}{\PYZsq{}}\PY{l+s+s1}{Número de passageiros no Titanic}\PY{l+s+s1}{\PYZsq{}}\PY{p}{)}
         \PY{n}{plt}\PY{o}{.}\PY{n}{subplot}\PY{p}{(}\PY{l+m+mi}{122}\PY{p}{)}
         \PY{n}{ax} \PY{o}{=} \PY{n}{plt}\PY{o}{.}\PY{n}{bar}\PY{p}{(}\PY{p}{[}\PY{l+m+mi}{0}\PY{p}{,} \PY{l+m+mi}{1}\PY{p}{,} \PY{l+m+mi}{2}\PY{p}{]}\PY{p}{,} \PY{n}{dados4}\PY{o}{.}\PY{n}{Pclass}\PY{o}{.}\PY{n}{value\PYZus{}counts}\PY{p}{(}\PY{n}{sort} \PY{o}{=} \PY{k+kc}{False}\PY{p}{)}\PY{p}{,} \PY{n}{color} \PY{o}{=} \PY{n}{tabela\PYZus{}cores}\PY{p}{)}
         \PY{n}{plt}\PY{o}{.}\PY{n}{xticks}\PY{p}{(}\PY{p}{[}\PY{l+m+mi}{0}\PY{p}{,} \PY{l+m+mi}{1}\PY{p}{,} \PY{l+m+mi}{2}\PY{p}{]}\PY{p}{,} \PY{p}{(}\PY{l+s+s1}{\PYZsq{}}\PY{l+s+s1}{1° Classe}\PY{l+s+s1}{\PYZsq{}}\PY{p}{,} \PY{l+s+s1}{\PYZsq{}}\PY{l+s+s1}{2° Classe}\PY{l+s+s1}{\PYZsq{}}\PY{p}{,} \PY{l+s+s1}{\PYZsq{}}\PY{l+s+s1}{3° Classe}\PY{l+s+s1}{\PYZsq{}}\PY{p}{)}\PY{p}{)}
         \PY{n}{plt}\PY{o}{.}\PY{n}{ylabel}\PY{p}{(}\PY{l+s+s1}{\PYZsq{}}\PY{l+s+s1}{Número de passageiros no Titanic}\PY{l+s+s1}{\PYZsq{}}\PY{p}{)}
         \PY{n}{sns}\PY{o}{.}\PY{n}{despine}\PY{p}{(}\PY{p}{)}
\end{Verbatim}


    \begin{center}
    \adjustimage{max size={0.9\linewidth}{0.9\paperheight}}{output_39_0.png}
    \end{center}
    { \hspace*{\fill} \\}
    
    É importante verificar a distribuição da idade dos passageiros.

    \begin{Verbatim}[commandchars=\\\{\}]
{\color{incolor}In [{\color{incolor}22}]:} \PY{c+c1}{\PYZsh{}Graficos da distribuicao de passageiros por idades}
         
         \PY{n}{plt}\PY{o}{.}\PY{n}{figure}\PY{p}{(}\PY{n}{figsize}\PY{o}{=}\PY{p}{(}\PY{l+m+mi}{10}\PY{p}{,}\PY{l+m+mi}{7}\PY{p}{)}\PY{p}{)}
         \PY{n}{ax} \PY{o}{=} \PY{n}{sns}\PY{o}{.}\PY{n}{distplot}\PY{p}{(}\PY{n}{dados4}\PY{p}{[}\PY{l+s+s1}{\PYZsq{}}\PY{l+s+s1}{Age}\PY{l+s+s1}{\PYZsq{}}\PY{p}{]}\PY{p}{,} \PY{n}{kde} \PY{o}{=} \PY{k+kc}{False}\PY{p}{,} \PY{n}{color} \PY{o}{=} \PY{l+s+s1}{\PYZsq{}}\PY{l+s+s1}{blue}\PY{l+s+s1}{\PYZsq{}}\PY{p}{,} \PY{n}{bins} \PY{o}{=} \PY{l+m+mi}{15}\PY{p}{)}
         \PY{n}{ax}\PY{o}{.}\PY{n}{set}\PY{p}{(}\PY{n}{xlabel}\PY{o}{=}\PY{l+s+s1}{\PYZsq{}}\PY{l+s+s1}{Idade}\PY{l+s+s1}{\PYZsq{}}\PY{p}{,} \PY{n}{ylabel} \PY{o}{=} \PY{l+s+s1}{\PYZsq{}}\PY{l+s+s1}{Número de passageiros no Titanic}\PY{l+s+s1}{\PYZsq{}}\PY{p}{)}
         \PY{n}{plt}\PY{o}{.}\PY{n}{show}\PY{p}{(}\PY{p}{)}
\end{Verbatim}


    \begin{center}
    \adjustimage{max size={0.9\linewidth}{0.9\paperheight}}{output_41_0.png}
    \end{center}
    { \hspace*{\fill} \\}
    
    Agora vamos começar a responder as questões do teste.

\begin{enumerate}
\def\labelenumi{\arabic{enumi})}
\tightlist
\item
  Existe diferença significativa entre as proporções de sobreviventes
  entre homens e mulheres?
\end{enumerate}

\paragraph{Resposta:}\label{resposta}

Primeiro calculamos as proporções de homens e mulheres sobreviventes.
Essa proporção é dada na tabela abaixo.

    \begin{Verbatim}[commandchars=\\\{\}]
{\color{incolor}In [{\color{incolor}23}]:} \PY{c+c1}{\PYZsh{}Sobreviventes por sexo}
         \PY{n}{dados4}\PY{p}{[}\PY{p}{[}\PY{l+s+s1}{\PYZsq{}}\PY{l+s+s1}{Sex}\PY{l+s+s1}{\PYZsq{}}\PY{p}{,}\PY{l+s+s1}{\PYZsq{}}\PY{l+s+s1}{Survived}\PY{l+s+s1}{\PYZsq{}}\PY{p}{]}\PY{p}{]}\PY{o}{.}\PY{n}{groupby}\PY{p}{(}\PY{p}{[}\PY{l+s+s1}{\PYZsq{}}\PY{l+s+s1}{Sex}\PY{l+s+s1}{\PYZsq{}}\PY{p}{]}\PY{p}{,} \PY{n}{as\PYZus{}index} \PY{o}{=} \PY{k+kc}{False}\PY{p}{)}\PY{o}{.}\PY{n}{mean}\PY{p}{(}\PY{p}{)}
\end{Verbatim}


\begin{Verbatim}[commandchars=\\\{\}]
{\color{outcolor}Out[{\color{outcolor}23}]:}    Sex  Survived
         0    0  0.742038
         1    1  0.188908
\end{Verbatim}
            
    O número de mulheres que sobreviveram é acentuadamente maior em relação
aos homens. Podemos avaliar a associação existente entre variáveis
qualitativas realizando o teste \(\chi^2\) "qui-quadrado". O princípio
básico deste teste é comparar proporções, ou seja, as possíveis
divergências entre as frequências observadas e esperadas para um evento.
A hipóteses que queremos testar são:

Hipótese nula: As frequências observadas não são diferentes das
frequências esperadas. Não existe diferença entre as frequências
(contagens) de sobreviventes por sexo. Assim, não existe associação
entre os grupos, sobreviventes por sexo.

Hipótese alternativa: As frequências observadas são diferentes das
frequências esperadas. Portanto existe diferença entre as frequências.
Assim, existe associação entre os grupos sobreviventes por sexo.

Para calcularmos o teste \(\chi^2\) criamos uma tabela com as
frequências de sobreviventes por sexo:

    \begin{Verbatim}[commandchars=\\\{\}]
{\color{incolor}In [{\color{incolor}24}]:} \PY{c+c1}{\PYZsh{}Tabela de contigencia}
         \PY{n}{dados4}\PY{p}{[}\PY{l+s+s1}{\PYZsq{}}\PY{l+s+s1}{N}\PY{l+s+s1}{\PYZsq{}}\PY{p}{]} \PY{o}{=} \PY{l+m+mi}{1}
         \PY{n}{tabela\PYZus{}sbys} \PY{o}{=} \PY{n}{pd}\PY{o}{.}\PY{n}{pivot\PYZus{}table}\PY{p}{(}\PY{n}{dados4}\PY{p}{,} \PY{n}{values} \PY{o}{=} \PY{l+s+s1}{\PYZsq{}}\PY{l+s+s1}{N}\PY{l+s+s1}{\PYZsq{}}\PY{p}{,} \PY{n}{index} \PY{o}{=} \PY{p}{[}\PY{l+s+s1}{\PYZsq{}}\PY{l+s+s1}{Sex}\PY{l+s+s1}{\PYZsq{}}\PY{p}{]}\PY{p}{,} \PY{n}{columns} \PY{o}{=}\PY{l+s+s1}{\PYZsq{}}\PY{l+s+s1}{Survived}\PY{l+s+s1}{\PYZsq{}}\PY{p}{,} \PY{n}{aggfunc} \PY{o}{=} \PY{n}{np}\PY{o}{.}\PY{n}{sum}\PY{p}{)}
         \PY{n}{tabela\PYZus{}sbys}
\end{Verbatim}


\begin{Verbatim}[commandchars=\\\{\}]
{\color{outcolor}Out[{\color{outcolor}24}]:} Survived    0    1
         Sex               
         0          81  233
         1         468  109
\end{Verbatim}
            
    \begin{Verbatim}[commandchars=\\\{\}]
{\color{incolor}In [{\color{incolor}25}]:} \PY{c+c1}{\PYZsh{}Teste X2 qui\PYZhy{}quadrado sobreviventes por sexo}
         \PY{n}{obs} \PY{o}{=} \PY{p}{[}\PY{p}{[}\PY{n}{tabela\PYZus{}sbys}\PY{p}{[}\PY{l+m+mi}{0}\PY{p}{]}\PY{p}{,}\PY{n}{tabela\PYZus{}sbys}\PY{p}{[}\PY{l+m+mi}{1}\PY{p}{]}\PY{p}{]}\PY{p}{]}
         \PY{n}{x2}\PY{p}{,} \PY{n}{p}\PY{p}{,} \PY{n}{dof}\PY{p}{,} \PY{n}{exp} \PY{o}{=} \PY{n}{stats}\PY{o}{.}\PY{n}{chi2\PYZus{}contingency}\PY{p}{(}\PY{n}{obs}\PY{p}{)}
         \PY{n}{x2}\PY{p}{,} \PY{n}{p}\PY{p}{,} \PY{n}{dof}    \PY{c+c1}{\PYZsh{}X2 calculado, p\PYZus{}valor, graus de liberdade}
\end{Verbatim}


\begin{Verbatim}[commandchars=\\\{\}]
{\color{outcolor}Out[{\color{outcolor}25}]:} (260.71702016732104, 1.1973570627755645e-58, 1)
\end{Verbatim}
            
    O valor do \(\chi^2\) foi de 260,72 com p-valor de 1,19\(e^{-58}\),
portanto, existe diferenças significativas entre os sobreviventes por
sexo, havendo influência do sexo em relação a sobrevivencia.

    \begin{enumerate}
\def\labelenumi{\arabic{enumi})}
\setcounter{enumi}{1}
\tightlist
\item
  Existe diferença significativa entre as proporções de sobreviventes
  entre classes diferentes?
\end{enumerate}

\paragraph{Resposta:}\label{resposta}

O raciocínio para resolução dessa questão é o mesmo da quetão anterior,
calculamos as proporções sobreviventes por classe social. Em seguida
faremos uma tabela de contigência e faremos o teste qui-quadrado.

    \begin{Verbatim}[commandchars=\\\{\}]
{\color{incolor}In [{\color{incolor}26}]:} \PY{c+c1}{\PYZsh{}Sobreviventes por classe social}
         \PY{n}{dados4}\PY{p}{[}\PY{p}{[}\PY{l+s+s1}{\PYZsq{}}\PY{l+s+s1}{Pclass}\PY{l+s+s1}{\PYZsq{}}\PY{p}{,}\PY{l+s+s1}{\PYZsq{}}\PY{l+s+s1}{Survived}\PY{l+s+s1}{\PYZsq{}}\PY{p}{]}\PY{p}{]}\PY{o}{.}\PY{n}{groupby}\PY{p}{(}\PY{p}{[}\PY{l+s+s1}{\PYZsq{}}\PY{l+s+s1}{Pclass}\PY{l+s+s1}{\PYZsq{}}\PY{p}{]}\PY{p}{,} \PY{n}{as\PYZus{}index} \PY{o}{=} \PY{k+kc}{False}\PY{p}{)}\PY{o}{.}\PY{n}{mean}\PY{p}{(}\PY{p}{)}
\end{Verbatim}


\begin{Verbatim}[commandchars=\\\{\}]
{\color{outcolor}Out[{\color{outcolor}26}]:}    Pclass  Survived
         0       1  0.629630
         1       2  0.472826
         2       3  0.242363
\end{Verbatim}
            
    \begin{Verbatim}[commandchars=\\\{\}]
{\color{incolor}In [{\color{incolor}27}]:} \PY{c+c1}{\PYZsh{}Tabela de contigencia}
         \PY{n}{tabela\PYZus{}sbyc} \PY{o}{=} \PY{n}{pd}\PY{o}{.}\PY{n}{pivot\PYZus{}table}\PY{p}{(}\PY{n}{dados4}\PY{p}{,} \PY{n}{values} \PY{o}{=} \PY{l+s+s1}{\PYZsq{}}\PY{l+s+s1}{N}\PY{l+s+s1}{\PYZsq{}}\PY{p}{,} \PY{n}{index} \PY{o}{=} \PY{p}{[}\PY{l+s+s1}{\PYZsq{}}\PY{l+s+s1}{Pclass}\PY{l+s+s1}{\PYZsq{}}\PY{p}{]}\PY{p}{,} \PY{n}{columns} \PY{o}{=}\PY{l+s+s1}{\PYZsq{}}\PY{l+s+s1}{Survived}\PY{l+s+s1}{\PYZsq{}}\PY{p}{,} \PY{n}{aggfunc} \PY{o}{=} \PY{n}{np}\PY{o}{.}\PY{n}{sum}\PY{p}{)}
         \PY{n}{tabela\PYZus{}sbyc}
\end{Verbatim}


\begin{Verbatim}[commandchars=\\\{\}]
{\color{outcolor}Out[{\color{outcolor}27}]:} Survived    0    1
         Pclass            
         1          80  136
         2          97   87
         3         372  119
\end{Verbatim}
            
    \begin{Verbatim}[commandchars=\\\{\}]
{\color{incolor}In [{\color{incolor}28}]:} \PY{c+c1}{\PYZsh{}Teste X2 qui\PYZhy{}quadrado sobreviventes por classe}
         \PY{n}{obs\PYZus{}c} \PY{o}{=} \PY{p}{[}\PY{p}{[}\PY{n}{tabela\PYZus{}sbyc}\PY{p}{[}\PY{l+m+mi}{0}\PY{p}{]}\PY{p}{,}\PY{n}{tabela\PYZus{}sbyc}\PY{p}{[}\PY{l+m+mi}{1}\PY{p}{]}\PY{p}{]}\PY{p}{]}
         \PY{n}{x2\PYZus{}c}\PY{p}{,} \PY{n}{p\PYZus{}c}\PY{p}{,} \PY{n}{dof\PYZus{}c}\PY{p}{,} \PY{n}{exp\PYZus{}c} \PY{o}{=} \PY{n}{stats}\PY{o}{.}\PY{n}{chi2\PYZus{}contingency}\PY{p}{(}\PY{n}{obs\PYZus{}c}\PY{p}{)}
         \PY{n}{x2\PYZus{}c}\PY{p}{,} \PY{n}{p\PYZus{}c}\PY{p}{,} \PY{n}{dof\PYZus{}c}    \PY{c+c1}{\PYZsh{}X2 calculado, p\PYZus{}valor, graus de liberdade}
\end{Verbatim}


\begin{Verbatim}[commandchars=\\\{\}]
{\color{outcolor}Out[{\color{outcolor}28}]:} (102.88898875696056, 4.5492517112987927e-23, 2)
\end{Verbatim}
            
    O valor do \(\chi^2\) foi de 102,89 com p-valor de 4,24\(e^{-23}\),
portanto, existe diferenças significativas entre os sobreviventes por
classe social, existindo influência da classe social em relação a
sobrevivencia.

Para uma interpretação mais interativa foi criado gráficos com relação
aos sobreviventes por sexo, classe social e por sexo e classe social.

    \begin{Verbatim}[commandchars=\\\{\}]
{\color{incolor}In [{\color{incolor}29}]:} \PY{c+c1}{\PYZsh{}Graficos de sobreviventes por sexo e classe social}
         
         \PY{n}{plt}\PY{o}{.}\PY{n}{subplots}\PY{p}{(}\PY{n}{figsize}\PY{o}{=}\PY{p}{(}\PY{p}{[}\PY{l+m+mi}{17}\PY{p}{,}\PY{l+m+mi}{6}\PY{p}{]}\PY{p}{)}\PY{p}{)}
         \PY{n}{plt}\PY{o}{.}\PY{n}{subplot}\PY{p}{(}\PY{l+m+mi}{131}\PY{p}{)}
         \PY{n}{ax} \PY{o}{=} \PY{n}{sns}\PY{o}{.}\PY{n}{barplot}\PY{p}{(}\PY{l+s+s1}{\PYZsq{}}\PY{l+s+s1}{Sex}\PY{l+s+s1}{\PYZsq{}}\PY{p}{,} \PY{l+s+s1}{\PYZsq{}}\PY{l+s+s1}{Survived}\PY{l+s+s1}{\PYZsq{}}\PY{p}{,} \PY{n}{data} \PY{o}{=} \PY{n}{dados}\PY{p}{,} \PY{n}{ci} \PY{o}{=} \PY{k+kc}{None}\PY{p}{,} \PY{n}{palette} \PY{o}{=} \PY{n}{tabela\PYZus{}cores}\PY{p}{)}
         \PY{n}{ax}\PY{o}{.}\PY{n}{set\PYZus{}xticklabels}\PY{p}{(}\PY{n}{labels} \PY{o}{=} \PY{p}{[}\PY{l+s+s1}{\PYZsq{}}\PY{l+s+s1}{Masculino}\PY{l+s+s1}{\PYZsq{}}\PY{p}{,} \PY{l+s+s1}{\PYZsq{}}\PY{l+s+s1}{Feminino}\PY{l+s+s1}{\PYZsq{}}\PY{p}{]}\PY{p}{)}
         \PY{n}{ax}\PY{o}{.}\PY{n}{set}\PY{p}{(}\PY{n}{xlabel}\PY{o}{=}\PY{l+s+s1}{\PYZsq{}}\PY{l+s+s1}{\PYZsq{}}\PY{p}{,} \PY{n}{ylabel}\PY{o}{=}\PY{l+s+s1}{\PYZsq{}}\PY{l+s+s1}{Proporção de Sobreviventes}\PY{l+s+s1}{\PYZsq{}}\PY{p}{)}
         \PY{n}{sns}\PY{o}{.}\PY{n}{despine}\PY{p}{(}\PY{p}{)}
         \PY{n}{plt}\PY{o}{.}\PY{n}{subplot}\PY{p}{(}\PY{l+m+mi}{132}\PY{p}{)}
         \PY{n}{ax} \PY{o}{=} \PY{n}{sns}\PY{o}{.}\PY{n}{barplot}\PY{p}{(}\PY{l+s+s1}{\PYZsq{}}\PY{l+s+s1}{Pclass}\PY{l+s+s1}{\PYZsq{}}\PY{p}{,} \PY{l+s+s1}{\PYZsq{}}\PY{l+s+s1}{Survived}\PY{l+s+s1}{\PYZsq{}}\PY{p}{,} \PY{n}{data} \PY{o}{=} \PY{n}{dados}\PY{p}{,} \PY{n}{ci} \PY{o}{=} \PY{k+kc}{None}\PY{p}{,} \PY{n}{palette} \PY{o}{=} \PY{n}{tabela\PYZus{}cores}\PY{p}{)}
         \PY{n}{ax}\PY{o}{.}\PY{n}{set\PYZus{}xticklabels}\PY{p}{(}\PY{n}{labels} \PY{o}{=} \PY{p}{[}\PY{l+s+s1}{\PYZsq{}}\PY{l+s+s1}{1° Classe}\PY{l+s+s1}{\PYZsq{}}\PY{p}{,} \PY{l+s+s1}{\PYZsq{}}\PY{l+s+s1}{2° Classe}\PY{l+s+s1}{\PYZsq{}}\PY{p}{,} \PY{l+s+s1}{\PYZsq{}}\PY{l+s+s1}{3° Classe}\PY{l+s+s1}{\PYZsq{}}\PY{p}{]}\PY{p}{)}
         \PY{n}{ax}\PY{o}{.}\PY{n}{set}\PY{p}{(}\PY{n}{xlabel}\PY{o}{=}\PY{l+s+s1}{\PYZsq{}}\PY{l+s+s1}{\PYZsq{}}\PY{p}{,} \PY{n}{ylabel}\PY{o}{=}\PY{l+s+s1}{\PYZsq{}}\PY{l+s+s1}{Proporção de Sobreviventes}\PY{l+s+s1}{\PYZsq{}}\PY{p}{)}
         \PY{n}{sns}\PY{o}{.}\PY{n}{despine}\PY{p}{(}\PY{p}{)}
         \PY{n}{plt}\PY{o}{.}\PY{n}{subplot}\PY{p}{(}\PY{l+m+mi}{133}\PY{p}{)}
         \PY{n}{ax} \PY{o}{=} \PY{n}{sns}\PY{o}{.}\PY{n}{barplot}\PY{p}{(}\PY{l+s+s1}{\PYZsq{}}\PY{l+s+s1}{Pclass}\PY{l+s+s1}{\PYZsq{}}\PY{p}{,} \PY{l+s+s1}{\PYZsq{}}\PY{l+s+s1}{Survived}\PY{l+s+s1}{\PYZsq{}}\PY{p}{,} \PY{l+s+s1}{\PYZsq{}}\PY{l+s+s1}{Sex}\PY{l+s+s1}{\PYZsq{}}\PY{p}{,} \PY{n}{data} \PY{o}{=} \PY{n}{dados}\PY{p}{,} \PY{n}{ci} \PY{o}{=}\PY{k+kc}{None}\PY{p}{,} \PY{n}{palette} \PY{o}{=} \PY{n}{tabela\PYZus{}cores}\PY{p}{)}
         \PY{n}{line1} \PY{o}{=} \PY{n}{mlines}\PY{o}{.}\PY{n}{Line2D}\PY{p}{(}\PY{p}{[}\PY{p}{]}\PY{p}{,} \PY{p}{[}\PY{p}{]}\PY{p}{,} \PY{n}{color}\PY{o}{=}\PY{l+s+s1}{\PYZsq{}}\PY{l+s+s1}{\PYZsh{}78C850}\PY{l+s+s1}{\PYZsq{}}\PY{p}{,} \PY{n}{label}\PY{o}{=}\PY{l+s+s1}{\PYZsq{}}\PY{l+s+s1}{Masculino}\PY{l+s+s1}{\PYZsq{}}\PY{p}{,} \PY{n}{linewidth}\PY{o}{=}\PY{l+m+mi}{3}\PY{p}{)}
         \PY{n}{line2} \PY{o}{=} \PY{n}{mlines}\PY{o}{.}\PY{n}{Line2D}\PY{p}{(}\PY{p}{[}\PY{p}{]}\PY{p}{,} \PY{p}{[}\PY{p}{]}\PY{p}{,} \PY{n}{color}\PY{o}{=}\PY{l+s+s1}{\PYZsq{}}\PY{l+s+s1}{\PYZsh{}F08030}\PY{l+s+s1}{\PYZsq{}}\PY{p}{,}  \PY{n}{label}\PY{o}{=}\PY{l+s+s1}{\PYZsq{}}\PY{l+s+s1}{Feminino}\PY{l+s+s1}{\PYZsq{}}\PY{p}{,} \PY{n}{linewidth}\PY{o}{=}\PY{l+m+mi}{3}\PY{p}{)}
         \PY{n}{plt}\PY{o}{.}\PY{n}{legend}\PY{p}{(}\PY{n}{ncol}\PY{o}{=}\PY{l+m+mi}{1}\PY{p}{,} \PY{n}{loc}\PY{o}{=}\PY{l+s+s2}{\PYZdq{}}\PY{l+s+s2}{best}\PY{l+s+s2}{\PYZdq{}}\PY{p}{,} \PY{n}{handles}\PY{o}{=}\PY{p}{[}\PY{n}{line1}\PY{p}{,} \PY{n}{line2}\PY{p}{]}\PY{p}{)}
         \PY{n}{ax}\PY{o}{.}\PY{n}{set\PYZus{}xticklabels}\PY{p}{(}\PY{n}{labels} \PY{o}{=} \PY{p}{[}\PY{l+s+s1}{\PYZsq{}}\PY{l+s+s1}{1° Classe}\PY{l+s+s1}{\PYZsq{}}\PY{p}{,} \PY{l+s+s1}{\PYZsq{}}\PY{l+s+s1}{2° Classe}\PY{l+s+s1}{\PYZsq{}}\PY{p}{,} \PY{l+s+s1}{\PYZsq{}}\PY{l+s+s1}{3° Classe}\PY{l+s+s1}{\PYZsq{}}\PY{p}{]}\PY{p}{)}
         \PY{n}{ax}\PY{o}{.}\PY{n}{set}\PY{p}{(}\PY{n}{xlabel} \PY{o}{=}\PY{l+s+s1}{\PYZsq{}}\PY{l+s+s1}{\PYZsq{}}\PY{p}{,} \PY{n}{ylabel}\PY{o}{=}\PY{l+s+s1}{\PYZsq{}}\PY{l+s+s1}{Proporção de Sobreviventes}\PY{l+s+s1}{\PYZsq{}}\PY{p}{)}
         \PY{n}{sns}\PY{o}{.}\PY{n}{despine}\PY{p}{(}\PY{p}{)}
         \PY{n}{plt}\PY{o}{.}\PY{n}{show}\PY{p}{(}\PY{p}{)}
\end{Verbatim}


    \begin{center}
    \adjustimage{max size={0.9\linewidth}{0.9\paperheight}}{output_53_0.png}
    \end{center}
    { \hspace*{\fill} \\}
    
    Pelo gráfico de sobreviventes por sexo, temos que mais de 74,20\% das
mulheres sobreviveram enquanto apenas 18,89\% dos homens sobrevivrem,
pelo teste qui-quadrado e considerando um nível de significância de 5\%
(p-valor) essa diferença é . Em relação a classe social, 62,96\% dos
passageiros da primeira classe sobreviveram enquanto que apenas 24,23\%
dos passageiros da 3° classe sobreviveram, é importante ressaltar que a
3° classe possui o maior número de passageiros. De forma geral as
mulheres possuem a maior chance de sobreviver independente da classe, no
entanto mulheres da primeira classe possuem o dobro de chances de
sobreviverem em relação à mulheres da 3° classe. Já, homens possuem
menor chance de sobrevivência em relação às mulheres, independente da
classe. No entanto, homens da 1° classe duas vezes mais chances de
sobreviverem que homens da 3° classe.

    \begin{enumerate}
\def\labelenumi{\arabic{enumi})}
\setcounter{enumi}{2}
\tightlist
\item
  Existe diferença significativa entre as proporções de sobreviventes
  entre faixas etárias diferentes? Quão mais velho você precisa ser para
  que você não saísse vivo do desastre?
\end{enumerate}

\paragraph{Resposta}\label{resposta}

A pirâmide etária definida pelo IBGE possui 21 classes, porém com esse
número de classes fica difícil visulaizar alguma informção de forma mais
simples. Segundo os estatísticos Moretin e Bussab o mínimo de 5 e o
máximo 15 classes é o mais indicado para o resumo de qualquer variável.
Dessa forma, utilizaremos 11 classes para resumir a faixa etária de
passageiros. Posteriomente, criamos a sequência com os intervalos das
classes e em seguida vamos criar uma nova coluna na nossa tabela com a
indicação da faxa etária de cada passageiro. As classes representam os
intervalos numéricos em que a variável quantitativa foi classificada. A
amplitude da classe é determinada por
\(\frac{max(Idade) - min(Idade)}{N. classes}\).

    \begin{Verbatim}[commandchars=\\\{\}]
{\color{incolor}In [{\color{incolor}30}]:} \PY{c+c1}{\PYZsh{}Definindo a sequencia de classes, a menor idade é 0,4 e a maior é 81 anos,}
         \PY{c+c1}{\PYZsh{}assim,o valores minimo e maximo do nosso intervalo sera 0 e 81.}
         \PY{c+c1}{\PYZsh{}Amplitude}
         
         \PY{n}{n} \PY{o}{=} \PY{l+m+mi}{10}
         \PY{n}{amp} \PY{o}{=} \PY{n+nb}{round}\PY{p}{(}\PY{p}{(}\PY{n}{dados4}\PY{p}{[}\PY{l+s+s1}{\PYZsq{}}\PY{l+s+s1}{Age}\PY{l+s+s1}{\PYZsq{}}\PY{p}{]}\PY{o}{.}\PY{n}{max}\PY{p}{(}\PY{p}{)} \PY{o}{\PYZhy{}} \PY{n}{dados4}\PY{p}{[}\PY{l+s+s1}{\PYZsq{}}\PY{l+s+s1}{Age}\PY{l+s+s1}{\PYZsq{}}\PY{p}{]}\PY{o}{.}\PY{n}{min}\PY{p}{(}\PY{p}{)}\PY{p}{)}\PY{o}{/}\PY{n}{n}\PY{p}{)}
         \PY{n}{intervalos} \PY{o}{=} \PY{n+nb}{list}\PY{p}{(}\PY{n+nb}{range}\PY{p}{(}\PY{l+m+mi}{0}\PY{p}{,} \PY{l+m+mi}{89}\PY{p}{,} \PY{n+nb}{int}\PY{p}{(}\PY{n}{amp}\PY{p}{)}\PY{p}{)}\PY{p}{)}
         
         \PY{c+c1}{\PYZsh{}Agrupando a idade pela faixa etaria}
         \PY{n}{dados4}\PY{p}{[}\PY{l+s+s1}{\PYZsq{}}\PY{l+s+s1}{FaixaEtaria}\PY{l+s+s1}{\PYZsq{}}\PY{p}{]} \PY{o}{=} \PY{n}{np}\PY{o}{.}\PY{n}{nan}
         
         \PY{k}{for} \PY{n}{i} \PY{o+ow}{in} \PY{n+nb}{range}\PY{p}{(}\PY{n+nb}{len}\PY{p}{(}\PY{n}{intervalos}\PY{p}{)}\PY{o}{\PYZhy{}}\PY{l+m+mi}{1}\PY{p}{)}\PY{p}{:}
             \PY{n}{dados4}\PY{o}{.}\PY{n}{loc}\PY{p}{[}\PY{p}{(}\PY{n}{dados4}\PY{p}{[}\PY{l+s+s1}{\PYZsq{}}\PY{l+s+s1}{Age}\PY{l+s+s1}{\PYZsq{}}\PY{p}{]} \PY{o}{\PYZgt{}}\PY{o}{=} \PY{n}{intervalos}\PY{p}{[}\PY{n}{i}\PY{p}{]}\PY{p}{)} \PY{o}{\PYZam{}} \PY{p}{(}\PY{n}{dados4}\PY{p}{[}\PY{l+s+s1}{\PYZsq{}}\PY{l+s+s1}{Age}\PY{l+s+s1}{\PYZsq{}}\PY{p}{]} \PY{o}{\PYZlt{}} \PY{n}{intervalos}\PY{p}{[}\PY{n}{i}\PY{o}{+}\PY{l+m+mi}{1}\PY{p}{]}\PY{p}{)}\PY{p}{,} \PY{l+s+s1}{\PYZsq{}}\PY{l+s+s1}{FaixaEtaria}\PY{l+s+s1}{\PYZsq{}}\PY{p}{]} \PY{o}{=} \PY{n}{i}  
\end{Verbatim}


    \begin{Verbatim}[commandchars=\\\{\}]
{\color{incolor}In [{\color{incolor}31}]:} \PY{c+c1}{\PYZsh{}Sobreviventes por faixa etária}
         \PY{n}{dados4}\PY{p}{[}\PY{p}{[}\PY{l+s+s1}{\PYZsq{}}\PY{l+s+s1}{FaixaEtaria}\PY{l+s+s1}{\PYZsq{}}\PY{p}{,}\PY{l+s+s1}{\PYZsq{}}\PY{l+s+s1}{Survived}\PY{l+s+s1}{\PYZsq{}}\PY{p}{]}\PY{p}{]}\PY{o}{.}\PY{n}{groupby}\PY{p}{(}\PY{p}{[}\PY{l+s+s1}{\PYZsq{}}\PY{l+s+s1}{FaixaEtaria}\PY{l+s+s1}{\PYZsq{}}\PY{p}{]}\PY{p}{,} \PY{n}{as\PYZus{}index} \PY{o}{=} \PY{k+kc}{False}\PY{p}{)}\PY{o}{.}\PY{n}{mean}\PY{p}{(}\PY{p}{)}
\end{Verbatim}


\begin{Verbatim}[commandchars=\\\{\}]
{\color{outcolor}Out[{\color{outcolor}31}]:}     FaixaEtaria  Survived
         0           0.0  0.627119
         1           1.0  0.428571
         2           2.0  0.361386
         3           3.0  0.344130
         4           4.0  0.401274
         5           5.0  0.311828
         6           6.0  0.474576
         7           7.0  0.423077
         8           8.0  0.000000
         9           9.0  0.000000
         10         10.0  1.000000
\end{Verbatim}
            
    \begin{Verbatim}[commandchars=\\\{\}]
{\color{incolor}In [{\color{incolor}32}]:} \PY{c+c1}{\PYZsh{}Graficos da ddensidade de sobreviventes por faixa etaria}
         
         \PY{n}{plt}\PY{o}{.}\PY{n}{figure}\PY{p}{(}\PY{n}{figsize}\PY{o}{=}\PY{p}{(}\PY{l+m+mi}{10}\PY{p}{,}\PY{l+m+mi}{7}\PY{p}{)}\PY{p}{)}
         \PY{n}{ax} \PY{o}{=} \PY{n}{sns}\PY{o}{.}\PY{n}{barplot}\PY{p}{(}\PY{l+s+s1}{\PYZsq{}}\PY{l+s+s1}{FaixaEtaria}\PY{l+s+s1}{\PYZsq{}}\PY{p}{,} \PY{l+s+s1}{\PYZsq{}}\PY{l+s+s1}{Survived}\PY{l+s+s1}{\PYZsq{}}\PY{p}{,} \PY{n}{data} \PY{o}{=} \PY{n}{dados4}\PY{p}{,} \PY{n}{ci} \PY{o}{=} \PY{k+kc}{None}\PY{p}{,} \PY{n}{palette} \PY{o}{=} \PY{n}{tabela\PYZus{}cores}\PY{p}{)}
         \PY{n}{ax}\PY{o}{.}\PY{n}{set\PYZus{}xticklabels}\PY{p}{(}\PY{n}{labels} \PY{o}{=} \PY{p}{[}\PY{l+s+s1}{\PYZsq{}}\PY{l+s+s1}{0 a 7}\PY{l+s+s1}{\PYZsq{}}\PY{p}{,} \PY{l+s+s1}{\PYZsq{}}\PY{l+s+s1}{8 a 15}\PY{l+s+s1}{\PYZsq{}}\PY{p}{,} \PY{l+s+s1}{\PYZsq{}}\PY{l+s+s1}{16 a 23}\PY{l+s+s1}{\PYZsq{}}\PY{p}{,} \PY{l+s+s1}{\PYZsq{}}\PY{l+s+s1}{14 a 31}\PY{l+s+s1}{\PYZsq{}}\PY{p}{,} \PY{l+s+s1}{\PYZsq{}}\PY{l+s+s1}{32 a 39}\PY{l+s+s1}{\PYZsq{}}\PY{p}{,} \PY{l+s+s1}{\PYZsq{}}\PY{l+s+s1}{40 a 47}\PY{l+s+s1}{\PYZsq{}}\PY{p}{,}
                                     \PY{l+s+s1}{\PYZsq{}}\PY{l+s+s1}{48 a 55}\PY{l+s+s1}{\PYZsq{}}\PY{p}{,} \PY{l+s+s1}{\PYZsq{}}\PY{l+s+s1}{56 a 63}\PY{l+s+s1}{\PYZsq{}}\PY{p}{,} \PY{l+s+s1}{\PYZsq{}}\PY{l+s+s1}{64 a 71}\PY{l+s+s1}{\PYZsq{}}\PY{p}{,} \PY{l+s+s1}{\PYZsq{}}\PY{l+s+s1}{72 a 79}\PY{l+s+s1}{\PYZsq{}}\PY{p}{,} \PY{l+s+s1}{\PYZsq{}}\PY{l+s+s1}{80 a 87}\PY{l+s+s1}{\PYZsq{}}\PY{p}{]}\PY{p}{)}
         \PY{n}{ax}\PY{o}{.}\PY{n}{set}\PY{p}{(}\PY{n}{xlabel}\PY{o}{=}\PY{l+s+s1}{\PYZsq{}}\PY{l+s+s1}{Idades}\PY{l+s+s1}{\PYZsq{}}\PY{p}{,} \PY{n}{ylabel}\PY{o}{=}\PY{l+s+s1}{\PYZsq{}}\PY{l+s+s1}{Proporção de Sobreviventes}\PY{l+s+s1}{\PYZsq{}}\PY{p}{)}
         \PY{n}{sns}\PY{o}{.}\PY{n}{despine}\PY{p}{(}\PY{p}{)}
         \PY{n}{plt}\PY{o}{.}\PY{n}{show}\PY{p}{(}\PY{p}{)}
\end{Verbatim}


    \begin{center}
    \adjustimage{max size={0.9\linewidth}{0.9\paperheight}}{output_58_0.png}
    \end{center}
    { \hspace*{\fill} \\}
    
    Vamos verificar se existe diferenças entre as faixas etárias, criamos
uma tabela de contigência com as frequências de sobreviventes e não
sobreviventes, poteriomente será realizado o teste \(\chi^2\).

    \begin{Verbatim}[commandchars=\\\{\}]
{\color{incolor}In [{\color{incolor}33}]:} \PY{c+c1}{\PYZsh{}Tabela de contigencia}
         \PY{n}{tabela\PYZus{}sbyf} \PY{o}{=} \PY{n}{pd}\PY{o}{.}\PY{n}{pivot\PYZus{}table}\PY{p}{(}\PY{n}{dados4}\PY{p}{,} \PY{n}{values} \PY{o}{=} \PY{l+s+s1}{\PYZsq{}}\PY{l+s+s1}{N}\PY{l+s+s1}{\PYZsq{}}\PY{p}{,} \PY{n}{index} \PY{o}{=} \PY{p}{[}\PY{l+s+s1}{\PYZsq{}}\PY{l+s+s1}{FaixaEtaria}\PY{l+s+s1}{\PYZsq{}}\PY{p}{]}\PY{p}{,} \PY{n}{columns} \PY{o}{=}\PY{l+s+s1}{\PYZsq{}}\PY{l+s+s1}{Survived}\PY{l+s+s1}{\PYZsq{}}\PY{p}{,} \PY{n}{aggfunc} \PY{o}{=} \PY{n}{np}\PY{o}{.}\PY{n}{sum}\PY{p}{)}
         
         \PY{n}{tabela\PYZus{}sbyf}\PY{p}{[}\PY{l+m+mi}{0}\PY{p}{]}\PY{o}{.}\PY{n}{fillna}\PY{p}{(}\PY{l+m+mi}{0}\PY{p}{,} \PY{n}{inplace} \PY{o}{=} \PY{k+kc}{True}\PY{p}{)}
         \PY{n}{tabela\PYZus{}sbyf}\PY{p}{[}\PY{l+m+mi}{1}\PY{p}{]}\PY{o}{.}\PY{n}{fillna}\PY{p}{(}\PY{l+m+mi}{0}\PY{p}{,} \PY{n}{inplace} \PY{o}{=} \PY{k+kc}{True}\PY{p}{)}
         \PY{n}{tabela\PYZus{}sbyf}
\end{Verbatim}


\begin{Verbatim}[commandchars=\\\{\}]
{\color{outcolor}Out[{\color{outcolor}33}]:} Survived         0     1
         FaixaEtaria             
         0.0           22.0  37.0
         1.0           20.0  15.0
         2.0          129.0  73.0
         3.0          162.0  85.0
         4.0           94.0  63.0
         5.0           64.0  29.0
         6.0           31.0  28.0
         7.0           15.0  11.0
         8.0           11.0   0.0
         9.0            1.0   0.0
         10.0           0.0   1.0
\end{Verbatim}
            
    \begin{Verbatim}[commandchars=\\\{\}]
{\color{incolor}In [{\color{incolor}34}]:} \PY{c+c1}{\PYZsh{}Teste X2 qui\PYZhy{}quadrado sobreviventes por faixa etaria}
         \PY{n}{obs\PYZus{}fe} \PY{o}{=} \PY{p}{[}\PY{p}{[}\PY{n}{tabela\PYZus{}sbyf}\PY{p}{[}\PY{l+m+mi}{0}\PY{p}{]}\PY{p}{,}\PY{n}{tabela\PYZus{}sbyf}\PY{p}{[}\PY{l+m+mi}{1}\PY{p}{]}\PY{p}{]}\PY{p}{]}
         \PY{n}{x2\PYZus{}fe}\PY{p}{,} \PY{n}{p\PYZus{}fe}\PY{p}{,} \PY{n}{dof\PYZus{}fe}\PY{p}{,} \PY{n}{exp\PYZus{}fe} \PY{o}{=} \PY{n}{stats}\PY{o}{.}\PY{n}{chi2\PYZus{}contingency}\PY{p}{(}\PY{n}{obs\PYZus{}fe}\PY{p}{)}
         \PY{n}{x2\PYZus{}fe}\PY{p}{,} \PY{n}{p\PYZus{}fe}\PY{p}{,} \PY{n}{dof\PYZus{}fe}    \PY{c+c1}{\PYZsh{}X2 calculado, p\PYZus{}valor, graus de liberdade}
\end{Verbatim}


\begin{Verbatim}[commandchars=\\\{\}]
{\color{outcolor}Out[{\color{outcolor}34}]:} (30.682802266334072, 0.00066182239153781959, 10)
\end{Verbatim}
            
    Dessa forma, verificamos que existe diferença significativa entre o
grupo de sobreviventes pelas diferentes faixas etárias, ou seja, a faixa
etária influencia nas chances de sobrevivência. Em relação há quanto
mais velho você precisa ser para ter menos chances de sobreviver, é
possível responder esta pergunta pelo gráfico da proporção de
sobreviventes por faixa etária. Por esse gráfico podemos visualizar que
a maior chance de sobreviver é de passageiros na classe de 0 a 7 anos,
logo a partir do limite dessa classe as chances de sobreviver diminuem,
cabe resaltar que a partir dos 64 anos de vida a probabilidade de
sobrevivência foi quase nula, há não ser pelo fato que na classe de 80 a
88 anos, 100\% dos passageiros sobreviveram, porém pela tabela de
contigência constatamos que existe apenas uma ocorrência para essa faixa
etária, ou seja, apenas uma pessoa acima de 64 anos de idade sobreviveu.
Logo, qualquer pessoa com menos de 64 anos de idade possui melhores
chances de sobreviver e com menos de 7 anos de idade essa probabilidade
aumenta para mais de 60\%.

\begin{enumerate}
\def\labelenumi{\arabic{enumi})}
\setcounter{enumi}{3}
\tightlist
\item
  Quais variáveis explicam melhor os dados? Explique quais testes e
  modelos foram utilizados em sua resposta.
\end{enumerate}

\paragraph{Resposta:}\label{resposta}

Em um primeiro instante, pode-se pensar que quanto maior o banco de
dados, representados por um volume elevado de variáveis descritivas de
observações, seja preferível para a explicação de um fenômeno. Contudo,
algumas variáveis não acrescentam nehuma informação adicional para
explicação do fenômeno, além de um grande número de variáveis aumentar a
dimensionalidade dos dados gerando alto custo computacional e as vezes
inserindo um viés ao modelo. A alta dimensionalidade, ito é, muitos
atributos, ou colunas falando de banco de dados é um fator crítico para
o desempenho de muitos algoritmos. Diasnte disso, é importante verificar
quais as variáveis que mais contribuem para o estudo. A correlação entre
as variáveis dependentes pode trazer essa informação, lembrando que,
variáveis que apresentarem baixa correlação podem ser excluídas do
modelo e variáveis com alta correlação entre si podem ser substituidas
uma pela outra para evitar o problema da autocorrelação. Cabe ressaltar
que quase todos os métodos da estatística classe para inferir
contribuição de variáveis são lineares, logo ao se trabalhar com
problemas não lineares esses métodos apresentam baixa eficiência. Uma
alternativa é fazer uso de modelos de machine learning como, árvores de
decisões, florestas aleátorias e redes neurais.

    \begin{Verbatim}[commandchars=\\\{\}]
{\color{incolor}In [{\color{incolor}44}]:} \PY{c+c1}{\PYZsh{}Dados}
         
         \PY{n}{dados5} \PY{o}{=} \PY{n}{dados4}\PY{p}{[}\PY{p}{[}\PY{l+s+s1}{\PYZsq{}}\PY{l+s+s1}{Survived}\PY{l+s+s1}{\PYZsq{}}\PY{p}{,} \PY{l+s+s1}{\PYZsq{}}\PY{l+s+s1}{Pclass}\PY{l+s+s1}{\PYZsq{}}\PY{p}{,} \PY{l+s+s1}{\PYZsq{}}\PY{l+s+s1}{Sex}\PY{l+s+s1}{\PYZsq{}}\PY{p}{,}\PY{l+s+s1}{\PYZsq{}}\PY{l+s+s1}{Age}\PY{l+s+s1}{\PYZsq{}}\PY{p}{,} \PY{l+s+s1}{\PYZsq{}}\PY{l+s+s1}{SibSp}\PY{l+s+s1}{\PYZsq{}}\PY{p}{,} \PY{l+s+s1}{\PYZsq{}}\PY{l+s+s1}{Parch}\PY{l+s+s1}{\PYZsq{}}\PY{p}{,} \PY{l+s+s1}{\PYZsq{}}\PY{l+s+s1}{Fare}\PY{l+s+s1}{\PYZsq{}}\PY{p}{,} \PY{l+s+s1}{\PYZsq{}}\PY{l+s+s1}{Embarked}\PY{l+s+s1}{\PYZsq{}}\PY{p}{,} 
                          \PY{l+s+s1}{\PYZsq{}}\PY{l+s+s1}{Family}\PY{l+s+s1}{\PYZsq{}}\PY{p}{,} \PY{l+s+s1}{\PYZsq{}}\PY{l+s+s1}{Title}\PY{l+s+s1}{\PYZsq{}}\PY{p}{,} \PY{l+s+s1}{\PYZsq{}}\PY{l+s+s1}{FaixaEtaria}\PY{l+s+s1}{\PYZsq{}}\PY{p}{]}\PY{p}{]}
         
         \PY{c+c1}{\PYZsh{}Definindo a variavel resposta (dependente) e as variáveis preditoras (independentes)}
         
         \PY{n}{y\PYZus{}target} \PY{o}{=} \PY{n}{dados5}\PY{o}{.}\PY{n}{values}\PY{p}{[}\PY{p}{:}\PY{p}{,} \PY{l+m+mi}{0}\PY{p}{]}
         \PY{n}{X\PYZus{}feature} \PY{o}{=} \PY{n}{dados5}\PY{o}{.}\PY{n}{values}\PY{p}{[}\PY{p}{:}\PY{p}{,} \PY{l+m+mi}{1}\PY{p}{:}\PY{p}{:}\PY{p}{]}
\end{Verbatim}


    \begin{Verbatim}[commandchars=\\\{\}]
{\color{incolor}In [{\color{incolor}42}]:} \PY{c+c1}{\PYZsh{}Matriz de correlacao}
         
         \PY{n}{matriz\PYZus{}corr} \PY{o}{=} \PY{n}{dados5}\PY{o}{.}\PY{n}{corr}\PY{p}{(}\PY{p}{)}
         \PY{n}{ax} \PY{o}{=} \PY{n}{plt}\PY{o}{.}\PY{n}{subplots}\PY{p}{(}\PY{n}{figsize}\PY{o}{=}\PY{p}{(}\PY{l+m+mi}{25}\PY{p}{,}\PY{l+m+mi}{16}\PY{p}{)}\PY{p}{)}
         \PY{n}{sns}\PY{o}{.}\PY{n}{plt}\PY{o}{.}\PY{n}{yticks}\PY{p}{(}\PY{n}{fontsize}\PY{o}{=}\PY{l+m+mi}{14}\PY{p}{)}
         \PY{n}{sns}\PY{o}{.}\PY{n}{plt}\PY{o}{.}\PY{n}{xticks}\PY{p}{(}\PY{n}{fontsize}\PY{o}{=}\PY{l+m+mi}{14}\PY{p}{)}
         \PY{n}{sns}\PY{o}{.}\PY{n}{heatmap}\PY{p}{(}\PY{n}{matriz\PYZus{}corr}\PY{p}{,} \PY{n}{cmap}\PY{o}{=}\PY{l+s+s1}{\PYZsq{}}\PY{l+s+s1}{viridis}\PY{l+s+s1}{\PYZsq{}}\PY{p}{,} \PY{n}{linewidths}\PY{o}{=}\PY{l+m+mf}{0.1}\PY{p}{,}\PY{n}{vmax}\PY{o}{=}\PY{l+m+mf}{1.0}\PY{p}{,} \PY{n}{square}\PY{o}{=}\PY{k+kc}{True}\PY{p}{,} \PY{n}{annot}\PY{o}{=}\PY{k+kc}{True}\PY{p}{)}
         \PY{n}{plt}\PY{o}{.}\PY{n}{show}\PY{p}{(}\PY{p}{)}
\end{Verbatim}


    \begin{center}
    \adjustimage{max size={0.9\linewidth}{0.9\paperheight}}{output_64_0.png}
    \end{center}
    { \hspace*{\fill} \\}
    
    A Matriz de Correlação permite calcular a correlação entre variáveis
através dos coeficiente de correlção de Pearson, Sperman ou Kendal. O
heatmap criado acima, indica o nível de correlação de cada variável em
uma graduação de cores, quanto mais amarela a variável, maior a
correlação. Esse gráfico é útil para detectar quais variáveis possuem
maior correlação com a variável respota e com as demais variáveis
independentes. Pelo gráfico as variáveis que possuem maior correlação
com a variável resposta são: Classe, Sexo, Fare e Embarked, logo essas
variáveis podem explicar melhor a sobrevivência. Já, as variáveis Age e
FaixaEtaria são autocorrelacionadas, bem como as variáveis Family, SibSp
e Parch. Em modelos de regressão essas variáveis devem ser substituídas
pela que apresentar maior correlação com a variável sobrevivência.

A matriz de correlação é um método linear, diante disso em problemas não
lineares ela pode não encontrar relção entre as variáveis. Para reverter
o problema da não linearidade foi utilizado o modelo Random Forest para
estimar a sobrevivência e classificar as variáveis mais importantes. As
metodologias baseadas em árvores de decisão são as mais utilizadas para
encontrar relação entre variáveis.

    \begin{Verbatim}[commandchars=\\\{\}]
{\color{incolor}In [{\color{incolor}37}]:} \PY{c+c1}{\PYZsh{}Definindo o modelo Random Forest}
         
         \PY{n}{modelo\PYZus{}rf1} \PY{o}{=} \PY{n}{RandomForestRegressor}\PY{p}{(}\PY{n}{n\PYZus{}estimators}\PY{o}{=}\PY{l+m+mi}{1000}\PY{p}{)}
         \PY{n}{modelo\PYZus{}rf1}\PY{o}{.}\PY{n}{fit}\PY{p}{(}\PY{n}{X\PYZus{}feature}\PY{p}{,} \PY{n}{y\PYZus{}target}\PY{p}{)}
         \PY{n+nb}{print}\PY{p}{(}\PY{n}{r2\PYZus{}score}\PY{p}{(}\PY{n}{y\PYZus{}target}\PY{p}{,} \PY{n}{modelo\PYZus{}rf1}\PY{o}{.}\PY{n}{predict}\PY{p}{(}\PY{n}{X\PYZus{}feature}\PY{p}{)}\PY{p}{)}\PY{p}{)}
         \PY{n}{mean\PYZus{}squared\PYZus{}error}\PY{p}{(}\PY{n}{y\PYZus{}target}\PY{p}{,} \PY{n}{modelo\PYZus{}rf1}\PY{o}{.}\PY{n}{predict}\PY{p}{(}\PY{n}{X\PYZus{}feature}\PY{p}{)}\PY{p}{)}
         \PY{n}{importances} \PY{o}{=} \PY{n}{modelo\PYZus{}rf1}\PY{o}{.}\PY{n}{feature\PYZus{}importances\PYZus{}}
         \PY{n}{std} \PY{o}{=} \PY{n}{np}\PY{o}{.}\PY{n}{std}\PY{p}{(}\PY{p}{[}\PY{n}{tree}\PY{o}{.}\PY{n}{feature\PYZus{}importances\PYZus{}} \PY{k}{for} \PY{n}{tree} \PY{o+ow}{in} \PY{n}{modelo\PYZus{}rf1}\PY{o}{.}\PY{n}{estimators\PYZus{}}\PY{p}{]}\PY{p}{,}
                      \PY{n}{axis}\PY{o}{=}\PY{l+m+mi}{0}\PY{p}{)}
         \PY{n}{indices} \PY{o}{=} \PY{n}{np}\PY{o}{.}\PY{n}{argsort}\PY{p}{(}\PY{n}{importances}\PY{p}{)}\PY{p}{[}\PY{p}{:}\PY{p}{:}\PY{o}{\PYZhy{}}\PY{l+m+mi}{1}\PY{p}{]}
\end{Verbatim}


    \begin{Verbatim}[commandchars=\\\{\}]
0.892298907404

    \end{Verbatim}

    \begin{Verbatim}[commandchars=\\\{\}]
{\color{incolor}In [{\color{incolor}43}]:} \PY{c+c1}{\PYZsh{} Grafico para importancia das variaveis}
         \PY{n}{plt}\PY{o}{.}\PY{n}{figure}\PY{p}{(}\PY{n}{figsize} \PY{o}{=} \PY{p}{(}\PY{p}{[}\PY{l+m+mi}{15}\PY{p}{,}\PY{l+m+mi}{8}\PY{p}{]}\PY{p}{)}\PY{p}{)}
         \PY{n}{plt}\PY{o}{.}\PY{n}{bar}\PY{p}{(}\PY{n+nb}{range}\PY{p}{(}\PY{n}{X\PYZus{}feature}\PY{o}{.}\PY{n}{shape}\PY{p}{[}\PY{l+m+mi}{1}\PY{p}{]}\PY{p}{)}\PY{p}{,} \PY{n}{importances}\PY{p}{[}\PY{n}{indices}\PY{p}{]}\PY{p}{,}
                \PY{n}{color}\PY{o}{=}\PY{n}{tabela\PYZus{}cores}\PY{p}{,} \PY{n}{align}\PY{o}{=}\PY{l+s+s2}{\PYZdq{}}\PY{l+s+s2}{center}\PY{l+s+s2}{\PYZdq{}}\PY{p}{)}
         \PY{n}{plt}\PY{o}{.}\PY{n}{xticks}\PY{p}{(}\PY{n+nb}{range}\PY{p}{(}\PY{n}{X\PYZus{}feature}\PY{o}{.}\PY{n}{shape}\PY{p}{[}\PY{l+m+mi}{1}\PY{p}{]}\PY{p}{)}\PY{p}{,} \PY{p}{[}\PY{l+s+s1}{\PYZsq{}}\PY{l+s+s1}{Sexo}\PY{l+s+s1}{\PYZsq{}}\PY{p}{,} \PY{l+s+s1}{\PYZsq{}}\PY{l+s+s1}{Idade}\PY{l+s+s1}{\PYZsq{}}\PY{p}{,} \PY{l+s+s1}{\PYZsq{}}\PY{l+s+s1}{Taxa}\PY{l+s+s1}{\PYZsq{}}\PY{p}{,} \PY{l+s+s1}{\PYZsq{}}\PY{l+s+s1}{Classe}\PY{l+s+s1}{\PYZsq{}}\PY{p}{,} \PY{l+s+s1}{\PYZsq{}}\PY{l+s+s1}{Família}\PY{l+s+s1}{\PYZsq{}}\PY{p}{,} \PY{l+s+s1}{\PYZsq{}}\PY{l+s+s1}{Faixa Etária}\PY{l+s+s1}{\PYZsq{}}\PY{p}{,} 
                                                \PY{l+s+s1}{\PYZsq{}}\PY{l+s+s1}{Irmãos/Conjugues}\PY{l+s+s1}{\PYZsq{}}\PY{p}{,} \PY{l+s+s1}{\PYZsq{}}\PY{l+s+s1}{Título}\PY{l+s+s1}{\PYZsq{}}\PY{p}{,}  \PY{l+s+s1}{\PYZsq{}}\PY{l+s+s1}{Porto}\PY{l+s+s1}{\PYZsq{}}\PY{p}{,} \PY{l+s+s1}{\PYZsq{}}\PY{l+s+s1}{Filhos/Pais}\PY{l+s+s1}{\PYZsq{}}\PY{p}{]}\PY{p}{)}
         \PY{n}{plt}\PY{o}{.}\PY{n}{xlim}\PY{p}{(}\PY{p}{[}\PY{o}{\PYZhy{}}\PY{l+m+mi}{1}\PY{p}{,} \PY{n}{X\PYZus{}feature}\PY{o}{.}\PY{n}{shape}\PY{p}{[}\PY{l+m+mi}{1}\PY{p}{]}\PY{p}{]}\PY{p}{)}
         \PY{n}{plt}\PY{o}{.}\PY{n}{show}\PY{p}{(}\PY{p}{)}
         
         \PY{c+c1}{\PYZsh{}[\PYZsq{}Sexo\PYZsq{}, \PYZsq{}Idade\PYZsq{}, \PYZsq{}Taxa\PYZsq{}, \PYZsq{}Classe\PYZsq{}, \PYZsq{}Família\PYZsq{}, \PYZsq{}Faixa Etária\PYZsq{}, \PYZsq{}Irmãos/Conjugues\PYZsq{}, \PYZsq{}Título\PYZsq{},  \PYZsq{}Porto\PYZsq{}, \PYZsq{}Filhos/Pais\PYZsq{}]}
         \PY{c+c1}{\PYZsh{}[\PYZsq{}Pclass\PYZsq{}, \PYZsq{}Sex\PYZsq{},\PYZsq{}Age\PYZsq{}, \PYZsq{}SibSp\PYZsq{}, \PYZsq{}Parch\PYZsq{}, \PYZsq{}Fare\PYZsq{}, \PYZsq{}Embarked\PYZsq{}, \PYZsq{}Family\PYZsq{}, \PYZsq{}Title\PYZsq{}, \PYZsq{}FaixaEtaria\PYZsq{}]}
\end{Verbatim}


    \begin{center}
    \adjustimage{max size={0.9\linewidth}{0.9\paperheight}}{output_67_0.png}
    \end{center}
    { \hspace*{\fill} \\}
    
    Com o modelo random forest classificamos as variáveis mais importantes.
Assim, Sexo, Idade, Taxa e Classe são as variáveis que possuem maior
informção sobre a variável sobrevivência. É importante destacar que, se
as variáveis independentes não são autocorrelacionas "o que gera
estimativa viesada em modelos de regressão", podemos utilizar todas as
variáveis ou apenas as mais importantes que os resultados são
semelhantes. Usar todas as variáveis, salvo a única restrição de
autocorrelação em modelos de regressão, não prejudica significativamente
o desempenho do modelo, apenas gera maior custo computacional, se o
custo computacional é um fator limitante, estimativas com as variáveis
mais importantes gera resultados estatísticamente semelhantes.

    \begin{enumerate}
\def\labelenumi{\arabic{enumi})}
\setcounter{enumi}{2}
\tightlist
\item
  Crie um modelo que defina a probabilidade de sobrevivência a partir
  das características de cada passageiro. Obs.: Siga uma metodologia que
  valide o modelo criado.
\end{enumerate}

\paragraph{Resposta:}\label{resposta}

O modelo clássico para este tipo de análise é o modelo de regressão
logística. Com ele podemos estimar a probalidade de alguém sobreviver de
acordo com as informações obtidas com as demais variáveis.

    \begin{Verbatim}[commandchars=\\\{\}]
{\color{incolor}In [{\color{incolor}66}]:} \PY{c+c1}{\PYZsh{}Dados}
         
         \PY{c+c1}{\PYZsh{}Todas as variaveis}
         \PY{n}{dados5} \PY{o}{=} \PY{n}{dados4}\PY{p}{[}\PY{p}{[}\PY{l+s+s1}{\PYZsq{}}\PY{l+s+s1}{Survived}\PY{l+s+s1}{\PYZsq{}}\PY{p}{,} \PY{l+s+s1}{\PYZsq{}}\PY{l+s+s1}{Pclass}\PY{l+s+s1}{\PYZsq{}}\PY{p}{,} \PY{l+s+s1}{\PYZsq{}}\PY{l+s+s1}{Sex}\PY{l+s+s1}{\PYZsq{}}\PY{p}{,}\PY{l+s+s1}{\PYZsq{}}\PY{l+s+s1}{Age}\PY{l+s+s1}{\PYZsq{}}\PY{p}{,} \PY{l+s+s1}{\PYZsq{}}\PY{l+s+s1}{SibSp}\PY{l+s+s1}{\PYZsq{}}\PY{p}{,} \PY{l+s+s1}{\PYZsq{}}\PY{l+s+s1}{Parch}\PY{l+s+s1}{\PYZsq{}}\PY{p}{,} \PY{l+s+s1}{\PYZsq{}}\PY{l+s+s1}{Fare}\PY{l+s+s1}{\PYZsq{}}\PY{p}{,} \PY{l+s+s1}{\PYZsq{}}\PY{l+s+s1}{Embarked}\PY{l+s+s1}{\PYZsq{}}\PY{p}{,} 
                          \PY{l+s+s1}{\PYZsq{}}\PY{l+s+s1}{Family}\PY{l+s+s1}{\PYZsq{}}\PY{p}{,} \PY{l+s+s1}{\PYZsq{}}\PY{l+s+s1}{Title}\PY{l+s+s1}{\PYZsq{}}\PY{p}{,} \PY{l+s+s1}{\PYZsq{}}\PY{l+s+s1}{FaixaEtaria}\PY{l+s+s1}{\PYZsq{}}\PY{p}{]}\PY{p}{]}
         
         
         \PY{c+c1}{\PYZsh{}Variaveis mais importantes pelo random forest}
         \PY{n}{dados6} \PY{o}{=} \PY{n}{dados4}\PY{p}{[}\PY{p}{[}\PY{l+s+s1}{\PYZsq{}}\PY{l+s+s1}{Survived}\PY{l+s+s1}{\PYZsq{}}\PY{p}{,} \PY{l+s+s1}{\PYZsq{}}\PY{l+s+s1}{Pclass}\PY{l+s+s1}{\PYZsq{}}\PY{p}{,} \PY{l+s+s1}{\PYZsq{}}\PY{l+s+s1}{Sex}\PY{l+s+s1}{\PYZsq{}}\PY{p}{,}\PY{l+s+s1}{\PYZsq{}}\PY{l+s+s1}{Age}\PY{l+s+s1}{\PYZsq{}}\PY{p}{,} \PY{l+s+s1}{\PYZsq{}}\PY{l+s+s1}{Fare}\PY{l+s+s1}{\PYZsq{}}\PY{p}{]}\PY{p}{]}
         
         
         \PY{c+c1}{\PYZsh{}Todas as variaveis}
         
         \PY{n}{y\PYZus{}target} \PY{o}{=} \PY{n}{dados5}\PY{o}{.}\PY{n}{values}\PY{p}{[}\PY{p}{:}\PY{p}{,} \PY{l+m+mi}{0}\PY{p}{]}
         \PY{n}{X\PYZus{}feature} \PY{o}{=} \PY{n}{dados5}\PY{o}{.}\PY{n}{values}\PY{p}{[}\PY{p}{:}\PY{p}{,} \PY{l+m+mi}{1}\PY{p}{:}\PY{p}{:}\PY{p}{]}
         
         \PY{c+c1}{\PYZsh{}Variaveis escolhida pelo Random forest}
         
         \PY{n}{y\PYZus{}target1} \PY{o}{=} \PY{n}{dados6}\PY{o}{.}\PY{n}{values}\PY{p}{[}\PY{p}{:}\PY{p}{,} \PY{l+m+mi}{0}\PY{p}{]}
         \PY{n}{X\PYZus{}feature1} \PY{o}{=} \PY{n}{dados6}\PY{o}{.}\PY{n}{values}\PY{p}{[}\PY{p}{:}\PY{p}{,} \PY{l+m+mi}{1}\PY{p}{:}\PY{p}{:}\PY{p}{]}
\end{Verbatim}


    \begin{Verbatim}[commandchars=\\\{\}]
{\color{incolor}In [{\color{incolor}67}]:} \PY{c+c1}{\PYZsh{}Regressao logistica com todas as variaveis}
         
         \PY{n}{modelo\PYZus{}lr} \PY{o}{=} \PY{n}{LogisticRegression}\PY{p}{(}\PY{n}{C}\PY{o}{=}\PY{l+m+mf}{1.}\PY{p}{,} \PY{n}{solver}\PY{o}{=}\PY{l+s+s1}{\PYZsq{}}\PY{l+s+s1}{lbfgs}\PY{l+s+s1}{\PYZsq{}}\PY{p}{)}
         \PY{n}{modelo\PYZus{}lr}\PY{o}{.}\PY{n}{fit}\PY{p}{(}\PY{n}{X\PYZus{}feature}\PY{p}{,} \PY{n}{y\PYZus{}target}\PY{p}{)}
         \PY{n+nb}{print}\PY{p}{(}\PY{n}{modelo\PYZus{}lr}\PY{o}{.}\PY{n}{score}\PY{p}{(}\PY{n}{X\PYZus{}feature}\PY{p}{,} \PY{n}{y\PYZus{}target}\PY{p}{)}\PY{p}{)}          \PY{c+c1}{\PYZsh{}Acuracia do modelo}
         \PY{n}{mean\PYZus{}squared\PYZus{}error}\PY{p}{(}\PY{n}{y\PYZus{}target}\PY{p}{,} \PY{n}{modelo\PYZus{}lr}\PY{o}{.}\PY{n}{predict}\PY{p}{(}\PY{n}{X\PYZus{}feature}\PY{p}{)}\PY{p}{)}
\end{Verbatim}


    \begin{Verbatim}[commandchars=\\\{\}]
0.808080808081

    \end{Verbatim}

\begin{Verbatim}[commandchars=\\\{\}]
{\color{outcolor}Out[{\color{outcolor}67}]:} 0.19191919191919191
\end{Verbatim}
            
    \begin{Verbatim}[commandchars=\\\{\}]
{\color{incolor}In [{\color{incolor}69}]:} \PY{c+c1}{\PYZsh{}Regressao logistica com as variaveis Sexo, Idade, Classe e Taxa}
         
         \PY{n}{modelo\PYZus{}lr1} \PY{o}{=} \PY{n}{LogisticRegression}\PY{p}{(}\PY{n}{C}\PY{o}{=}\PY{l+m+mf}{1.}\PY{p}{,} \PY{n}{solver}\PY{o}{=}\PY{l+s+s1}{\PYZsq{}}\PY{l+s+s1}{lbfgs}\PY{l+s+s1}{\PYZsq{}}\PY{p}{)}
         \PY{n}{modelo\PYZus{}lr1}\PY{o}{.}\PY{n}{fit}\PY{p}{(}\PY{n}{X\PYZus{}feature1}\PY{p}{,} \PY{n}{y\PYZus{}target1}\PY{p}{)}
         \PY{n+nb}{print}\PY{p}{(}\PY{n}{modelo\PYZus{}lr1}\PY{o}{.}\PY{n}{score}\PY{p}{(}\PY{n}{X\PYZus{}feature1}\PY{p}{,} \PY{n}{y\PYZus{}target1}\PY{p}{)}\PY{p}{)}          \PY{c+c1}{\PYZsh{}Acuracia do modelo}
         \PY{n}{mean\PYZus{}squared\PYZus{}error}\PY{p}{(}\PY{n}{y\PYZus{}target1}\PY{p}{,} \PY{n}{modelo\PYZus{}lr1}\PY{o}{.}\PY{n}{predict}\PY{p}{(}\PY{n}{X\PYZus{}feature1}\PY{p}{)}\PY{p}{)}
\end{Verbatim}


    \begin{Verbatim}[commandchars=\\\{\}]
0.792368125701

    \end{Verbatim}

\begin{Verbatim}[commandchars=\\\{\}]
{\color{outcolor}Out[{\color{outcolor}69}]:} 0.20763187429854096
\end{Verbatim}
            
    O resultados com ambos os modelos logísticos são semelhantes, daí
podemos concluir que com apenas as quatros variáveis podemos obter
resultados semelhantes e no caso desse modelo de regressão não caímos no
problema de autoccorrelação de variáveis. A acurácia do modelo é de
aproximadamente 80\%, ou seja, o modelo está bem ajustado.

Como não exite preocupação com o custo computacional podemos treinar o
algoritmo random forest com todas as variáveis e verificar o seu
desenpenho.

    \begin{Verbatim}[commandchars=\\\{\}]
{\color{incolor}In [{\color{incolor}52}]:} \PY{n}{modelo\PYZus{}rf2} \PY{o}{=} \PY{n}{RandomForestRegressor}\PY{p}{(}\PY{n}{n\PYZus{}estimators}\PY{o}{=}\PY{l+m+mi}{1000}\PY{p}{)}
         \PY{n}{modelo\PYZus{}rf2}\PY{o}{.}\PY{n}{fit}\PY{p}{(}\PY{n}{X\PYZus{}feature2}\PY{p}{,} \PY{n}{y\PYZus{}target2}\PY{p}{)}
         \PY{n+nb}{print}\PY{p}{(}\PY{n}{r2\PYZus{}score}\PY{p}{(}\PY{n}{y\PYZus{}target2}\PY{p}{,} \PY{n}{modelo\PYZus{}rf2}\PY{o}{.}\PY{n}{predict}\PY{p}{(}\PY{n}{X\PYZus{}feature2}\PY{p}{)}\PY{p}{)}\PY{p}{)}
         \PY{n}{mean\PYZus{}squared\PYZus{}error}\PY{p}{(}\PY{n}{y\PYZus{}target2}\PY{p}{,} \PY{n}{modelo\PYZus{}rf2}\PY{o}{.}\PY{n}{predict}\PY{p}{(}\PY{n}{X\PYZus{}feature2}\PY{p}{)}\PY{p}{)}
\end{Verbatim}


    \begin{Verbatim}[commandchars=\\\{\}]
0.885175172306

    \end{Verbatim}

\begin{Verbatim}[commandchars=\\\{\}]
{\color{outcolor}Out[{\color{outcolor}52}]:} 0.027156815691676252
\end{Verbatim}
            
    Todos os modelos apresentam resultados estatisticamente significantes,
porém o modelo com menor erro quadrado médio é o mais indicado para
realizar estimativas. Um ponto positivo do modelo logístico é que é
possível facilmente estimar a probabilidade de sobrevivência de um
passageiro.

    \begin{enumerate}
\def\labelenumi{\arabic{enumi})}
\setcounter{enumi}{3}
\tightlist
\item
  Bônus: Qual probabilidade de um homem solteiro de 19 anos que embarcou
  em Southampton sozinho na terceira classe sobreviva ao desastre?
\end{enumerate}

\paragraph{Resposta:}\label{resposta}

    \begin{Verbatim}[commandchars=\\\{\}]
{\color{incolor}In [{\color{incolor}70}]:} \PY{c+c1}{\PYZsh{}Calculando a probalidade}
         \PY{c+c1}{\PYZsh{}Lembrando a ordem das informacoes de X}
         \PY{c+c1}{\PYZsh{}Codificacoes}
         \PY{c+c1}{\PYZsh{}Sex}
         \PY{c+c1}{\PYZsh{}female = 0, male = 1,}
         \PY{c+c1}{\PYZsh{}Embarked}
         \PY{c+c1}{\PYZsh{}C = 0, Q = 1, S = 2}
         \PY{c+c1}{\PYZsh{}Title}
         \PY{c+c1}{\PYZsh{}Master = 0, Miss = 1, Mr = 2, Mrs = 3, Rich = 4}
         \PY{c+c1}{\PYZsh{}[\PYZsq{}Pclass\PYZsq{}, \PYZsq{}Sex\PYZsq{},\PYZsq{}Age\PYZsq{}, \PYZsq{}SibSp\PYZsq{}, \PYZsq{}Parch\PYZsq{}, \PYZsq{}Fare\PYZsq{}, \PYZsq{}Embarked\PYZsq{}, \PYZsq{}Family\PYZsq{}, \PYZsq{}Title\PYZsq{}, \PYZsq{}FaixaEtaria\PYZsq{}]}
         
         \PY{n}{X\PYZus{}novo} \PY{o}{=} \PY{n}{np}\PY{o}{.}\PY{n}{array}\PY{p}{(}\PY{p}{[}\PY{l+m+mf}{3.}\PY{p}{,} \PY{l+m+mf}{1.}\PY{p}{,} \PY{l+m+mf}{19.}\PY{p}{,} \PY{l+m+mf}{0.}\PY{p}{]}\PY{p}{)}
         \PY{n}{modelo\PYZus{}lr1}\PY{o}{.}\PY{n}{predict\PYZus{}proba}\PY{p}{(}\PY{n}{X\PYZus{}novo}\PY{p}{)}
\end{Verbatim}


    \begin{Verbatim}[commandchars=\\\{\}]
/usr/local/lib/python3.5/dist-packages/sklearn/utils/validation.py:395: DeprecationWarning: Passing 1d arrays as data is deprecated in 0.17 and will raise ValueError in 0.19. Reshape your data either using X.reshape(-1, 1) if your data has a single feature or X.reshape(1, -1) if it contains a single sample.
  DeprecationWarning)

    \end{Verbatim}

\begin{Verbatim}[commandchars=\\\{\}]
{\color{outcolor}Out[{\color{outcolor}70}]:} array([[ 0.87950816,  0.12049184]])
\end{Verbatim}
            
    Logo, um passageiro com essas características tem probalidade de
sobrevivência de 12,04\%.


    % Add a bibliography block to the postdoc
    
    
    
    \end{document}
